\section{Cyphal/serial (experimental)}\label{sec:transport_serial}

\hyphenation{Cyphal/serial}  % Disable hyphenation.

\subsection{Overview}

This section specifies a concrete transport that operates on top of raw byte-level communication channels,
such as TCP/IP connections, SSL, UART, RS-232, RS-422, USB CDC ACM,
and any similar communication links that allow exchange of unstructured byte streams.
Cyphal/serial may also be used to store Cyphal frames in files.
\textbf{%
    As of this version, the Cyphal/serial specification remains experimental.
    Breaking changes affecting wire compatibility are possible.
}

As Cyphal/serial is designed to operate over unstructured byte streams,
it defines a custom framing protocol, custom frame header format, and a custom integrity checking mechanism.

\begin{CyphalSimpleTable}{Cyphal/serial transport capabilities\label{table:transport_serial_capabilities}}{|l X l|}
    Parameter & Value & References \\

    Maximum node-ID value &
    65534 (16 bits wide). &
    \ref{sec:basic} \\

    Transfer-ID mode &
    Monotonic, 64 bits wide. &
    \ref{sec:transport_transfer_id} \\

    Number of transfer priority levels &
    8 (no additional levels). &
    \ref{sec:transport_transfer_priority} \\

    Largest single-frame transfer payload &
    Unlimited. &
    \ref{sec:transport_transfer_payload} \\

    Anonymous transfers &
    Available. &
    \ref{sec:transport_route_specifier} \\
\end{CyphalSimpleTable}

\subsection{Framing}

Cyphal/serial uses the ``Consistent Overhead Byte Stuffing'' (COBS) encapsulation method\footnote{%
    Stuart Cheshire and Mary Baker. 1999. Consistent overhead Byte stuffing.
    IEEE/ACM Trans. Netw. 7, 2 (April 1999), 159--172. \mbox{\url{https://doi.org/10.1109/90.769765}}.
    The COBS overhead is 1 byte in every 254 bytes of encapsulated data, which is about 0.4\%.
} with zero byte as the frame delimiter.
Due to the nature of COBS, the frame delimiter will not appear in the frame payload.
A frame delimiter may terminate a frame and/or indicate the start of a new frame.
The number of frame delimiters between adjacent frames may exceed one.

\begin{figure}[H]
    \centering
    $$
    \texttt{\huge{...}}%
    \underbrace{\texttt{\huge{0}}}_{\substack{\text{frame} \\ \text{delimiter}}}%
    \underbrace{\texttt{\huge{<frame>}}}_{\substack{\text{COBS-encoded} \\ \text{frame contents}}}%
    \underbrace{\texttt{\huge{0}}}_{\substack{\text{frame} \\ \text{delimiter}}}%
    \underbrace{\texttt{\huge{<frame>}}}_{\substack{\text{COBS-encoded} \\ \text{frame contents}}}%
    \underbrace{\texttt{\huge{0}}}_{\substack{\text{frame} \\ \text{delimiter}}}%
    \texttt{\huge{...}}%
    $$
    \caption{COBS framing\label{fig:transport_serial_cobs}}
\end{figure}

A frame consists of two parts:
the fixed-size header (section~\ref{sec:transport_serial_header})
immediately followed by the payload (section~\ref{sec:transport_serial_payload}).
The header contains a dedicated header-CRC field which allows implementations to detect frame corruption early.

Neither the underlying medium nor the Cyphal/serial transport layer impose any restrictions on the maximum frame size,
which allows all Cyphal/serial transfers to be single-frame transfers\footnote{%
    Omitting multi-frame transfers as a requirement is expected to simplify implementations.
}.

\subsection{Header}\label{sec:transport_serial_header}

The layout of the Cyphal/serial header is shown in the following snippet in DSDL notation
(section~\ref{sec:dsdl}).

\begin{samepage}
\begin{minted}{python}
# This 24-byte header can be aliased as a C structure with each field being naturally aligned:
#
#       uint8_t  version;
#       uint8_t  priority;
#       uint16_t source_node_id;
#       uint16_t destination_node_id;
#       uint16_t data_specifier_snm;
#       uint64_t transfer_id;
#       uint32_t frame_index_eot;
#       uint16_t user_data;
#       uint8_t  header_crc16_big_endian[2];

uint4 version
# The version of the header format. This document specifies version 1.
# Packets with an unknown version number must be ignored.

void4

uint3 priority
# The values are assigned from 0 (HIGHEST priority) to 7 (LOWEST priority).
# The numerical priority identifiers are chosen to be consistent with Cyphal/CAN.

void5

uint16 source_node_id
# The node-ID of the source node.
# Value 65535 represents anonymous transfers.

uint16 destination_node_id
# The node-ID of the destination node.
# Value 65535 represents broadcast transfers.

uint15 data_specifier
# If this is a message transfer, this value equals the subject-ID.
# If this is a service response transfer, this value equals the service-ID.
# If this is a service request transfer, this value equals 16384 + service-ID.

bool service_not_message
# If true, this is a service transfer. If false, this is a message transfer.

@assert _offset_ == {64}
uint64 transfer_id
# The monotonic transfer-ID value of the current transfer (never overflows).

uint31 frame_index
# Transmit zero. Drop frame if received non-zero.

bool end_of_transfer
# Transmit true. Drop frame if received false.

uint16 user_data
# Opaque application-specific data with user-defined semantics.
# Generic implementations should emit zero and ignore this field upon reception.

uint8[2] header_crc16_big_endian
# CRC-16/CCITT-FALSE of the preceding serialized header data in the big endian byte order.
# Application of the CRC function to the entire header shall yield zero, otherwise the header is malformed.

@assert _offset_ / 8 == {24}
@sealed     # The payload data follows.
\end{minted}
\end{samepage}

\subsection{Payload}\label{sec:transport_serial_payload}

The transfer payload is appended with a transfer CRC field.
The transfer CRC function is \textbf{CRC-32C} (section~\ref{sec:appendix_crc32c}),
and its value is serialized in the little-endian byte order.
The transfer CRC function is applied to the entire transfer payload and only transfer payload (header not included).

A node receiving a transfer should verify the correctness of its transfer CRC.

\subsection{Examples}

% $ ncat --broker --listen -p 50905 -vv
%
% $ nc localhost 50905 | xxd -g1
%
% $ export UAVCAN__SERIAL__IFACE='socket://127.0.0.1:50905'
% $ export UAVCAN__NODE__ID=1234
% $ y pub -N1 1234:uavcan.primitive.string '"012345678"'
\begin{remark}
    The snippet given below contains the hexadecimal dump of the following Cyphal/serial transfer:

    \begin{description}
        \item[Priority] nominal
        \item[Transfer-ID] 0
        \item[Transfer kind] message with the subject-ID 1234
        \item[Source node-ID] 1234
        \item[Destination node-ID] None
        \item[Header user data] 0
        \item[Transfer payload] \verb|uavcan.primitive.String.1| containing string ``\verb|012345678|''
    \end{description}

    The payload is shown in segments for clarity:

    \begin{itemize}
        \item The first byte is the starting delimiter of the first frame.
        \item The second byte is a COBS overhead byte (one for the entire transfer).
        \item The following block of 24 bytes is the COBS-encoded header.
        \item The third-to-last block is the COBS-encoded transfer payload,
        containing the two bytes of the array length prefix followed by the string data.
        \item The second-to-last block of four bytes is the COBS-encoded transfer-CRC.
        \item The last byte is the ending frame delimiter.
    \end{itemize}

    \begin{verbatim}
00
09
01 04 d2 04 ff ff d2 04 01 01 01 01 01 01 01 01 01 01 02 80 01 04 08 12
09 0e 30 31 32 33 34 35 36 37 38
84 a2 2d e2
00
    \end{verbatim}
\end{remark}

\begin{remark}
    The snippet given below contains the hexadecimal dump of the following Cyphal/serial transfer:

    \begin{description}
        \item[Priority] nominal
        \item[Transfer-ID] 0
        \item[Transfer kind] message with the subject-ID 1234
        \item[Source node-ID] 4321
        \item[Destination node-ID] None
        \item[Header user data] 0
        \item[Transfer payload] \verb|uavcan.primitive.Empty.1|
    \end{description}

    The payload is shown in segments for clarity:

    \begin{itemize}
        \item The first byte is the starting delimiter of the first frame.
        \item The second byte is a COBS overhead byte (one for the entire transfer).
        \item The following block of 24 bytes is the COBS-encoded header.
        \item The second-to-last block of four bytes is the COBS-encoded transfer-CRC,
        which is zero as the payload is empty.
        \item The last byte is the ending frame delimiter.
    \end{itemize}

    \begin{verbatim}
00
09
01 04 e1 10 ff ff d2 04 01 01 01 01 01 01 01 01 01 01 02 80 01 03 93 70
01 01 01 01
00
    \end{verbatim}
\end{remark}

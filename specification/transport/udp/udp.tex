\section{Cyphal/UDP (experimental)}\label{sec:transport_udp}

\hyphenation{Cyphal/UDP}  % Disable hyphenation.

\subsection{Overview}

This section specifies a concrete transport based on the UDP/IPv4 protocol\footnote{%
    Support for IPv6 may appear in future versions of this specification.
}, as specified in IETF RFC~768.
\textbf{
    As of this version, the Cyphal/UDP specification remains experimental.
    Breaking changes affecting wire compatibility are possible.
}

Cyphal/UDP is a first-class transport intended for low-latency,
high-throughput intravehicular Ethernet networks with complex topologies,
which may be switched, multi-drop, or mixed.
A network utilizing Cyphal/UDP can be built with standards-compliant commercial off-the-shelf
networking equipment and software.

Cyphal/UDP relies exclusively on IP multicast traffic defined in IETF RFC~1112 for all communication\footnote{%
    For rationale, refer to \url{https://forum.opencyphal.org/t/1765}.
}.
The entirety of the session specifier (section~\ref{sec:transport_session_specifier})
is reified through the multicast group address.
The transfer-ID, transfer priority, and the multi-frame transfer reassembly metadata are allocated in the
Cyphal-specific fixed-size UDP datagram header.
In this transport, a UDP datagram represents a single Cyphal transport frame.
All UDP datagrams are addressed to the same, fixed, destination port,
while the source port and the source address bear no relevance for the protocol and thus can be arbitrary.

\begin{CyphalSimpleTable}{Cyphal/UDP transport capabilities\label{table:transport_udp_capabilities}}{|l X l|}
    Parameter & Value & References \\

    Maximum node-ID value &
    65534 (16 bits wide). &
    \ref{sec:basic} \\

    Transfer-ID mode &
    Monotonic, 64 bits wide. &
    \ref{sec:transport_transfer_id} \\

    Number of transfer priority levels &
    8 (no additional levels). &
    \ref{sec:transport_transfer_priority} \\

    Largest single-frame transfer payload &
    % 480 bytes = 508 bytes minus 24 bytes for the Cyphal/UDP header minus 4 bytes for the transfer CRC.
    % 65479 bytes = 65507 bytes minus 24 bytes for the Cyphal/UDP header minus 4 bytes for the transfer CRC.
    Implementation-defined, but not less than 480~bytes and not greater than 65479~bytes. &
    \ref{sec:transport_transfer_payload} \\

    Anonymous transfers &
    Available. &
    \ref{sec:transport_route_specifier} \\
\end{CyphalSimpleTable}

\subsection{UDP/IP endpoints and routing}

\subsubsection{Endpoints}

Transmission of a Cyphal/UDP transport frame is performed by sending a suitably constructed UDP datagram
to the destination IP multicast group address computed from the session specifier
(section~\ref{sec:transport_session_specifier})
as shown in figure~\ref{fig:transport_udp_multicast_group_address}
with the fixed destination port number \textbf{9382}\footnote{%
    % Update this footnote when the EXPERIMENTAL status is lifted.
    The port number may change if/when the protocol is registered with IANA,
    unless it is stabilized in its current form.
}.

\begin{figure}[H]
    \centering
    $$
    \overbrace{
        \underbrace{
            \texttt{\huge{1110}}
        }_{\substack{\text{RFC~1112} \\ \text{multicast} \\ \text{prefix}}}%
        \underbrace{
            \texttt{\huge{1111}}
        }_{\substack{\text{RFC~2365} \\ \text{administrative} \\ \text{scope}}}%
    }^{\text{Most significant octet}}%
    \texttt{\huge{.}}% ----------------------------------------
    \overbrace{
        \underbrace{
            \texttt{\huge{0}}
        }_{\substack{\text{RFC~2365} \\ \text{reserved} \\ \text{range}}}%
        \underbrace{
            \texttt{\huge{0}}
        }_{\substack{\text{address} \\ \text{version}}}%
        \underbrace{
            \texttt{\huge{00000}}
        }_{\substack{\text{reserved} \\ \text{keep zero}}}%
        \underbrace{
            \texttt{\huge{Z}}
        }_{\substack{\text{service,} \\ \text{not} \\ \text{message}}}%
    }^{\text{3rd octet}}%
    \texttt{\huge{.}}% ----------------------------------------
    \underbrace{
        \overbrace{\texttt{\huge{XXXXXXXX}}}^{\text{2nd octet}}
        \texttt{\huge{.}}
        \overbrace{\texttt{\huge{XXXXXXXX}}}^{\text{Least significant octet}}
    }_{\substack{\text{\textbf{if Z:} destination node-ID} \\ \text{\textbf{else:} subject-ID + reserved}}}%
    $$
    Numbers given in base-2.
    \caption{IP multicast group address structure\label{fig:transport_udp_multicast_group_address}}
\end{figure}

\begin{CyphalSimpleTable}[wide]{
    IP multicast group address bit fields\label{table:transport_udp_multicast_group_address}
}{|l l l l X|}
    Field & Offset & Width & Value & Description \\

    RFC~1112 multicast prefix &
    28 & 4 & $1110_2$ &
    \\

    RFC~2365 scope &
    24 & 4 & $1111_2$ &
    Selects the administratively scoped range 239.0.0.0/8 per RFC~2365
    to avoid collisions with well-known multicast groups. \\

    RFC~2365 reserved range &
    23 & 1 & $0$ &
    Selects the ad-hoc defined range 239.0.0.0/9 per RFC~2365. \\

    Cyphal/UDP address version &
    22 & 1 & $0$ &
    Deconflicts this layout with future revisions. \\

    Reserved &
    17 & 5 & $00000_2$ &
    May be used for domain-ID segregation in future versions. \\

    Z: service, not message &
    16 & 1 & any &
    Set for service transfers, cleared for message transfers. \\

    X if Z: destination node-ID &
    0 & 16 & $[0, 65534]$ &
    The destination node-ID of the current service transfer. \\

    X if not Z: reserved &
    13 & 3 & $0$ &
    May be used to enlarge the subject-ID field in future versions. \\

    X if not Z: subject-ID &
    0 & 13 & any &
    The subject-ID of the current message transfer. \\
\end{CyphalSimpleTable}

\begin{remark}
    Freezing (at least) the 9 most significant bits of the multicast group address ensures that
    the variability is confined to the 23 least significant bits of the address only,
    which is desirable because the IPv4 Ethernet MAC layer does not differentiate beyond the
    23 least significant bits of the multicast group address.
    That is, addresses that differ only in the 9 MSb collide at the MAC layer,
    which is unacceptable in a real-time system; see RFC~1112 section 6.4.
    Without this limitation, an engineer designing a network might inadvertently create a configuration
    that causes MAC-layer collisions which may be difficult to detect.
\end{remark}

A subscriber to certain Cyphal subjects will join the IP multicast groups corresponding to said subjects\footnote{%
    For example, the multicast group address for subject 42 is 239.0.0.42.
}.
Likewise, a node that provides at least one RPC-service will join the IP multicast group corresponding to
its own node-ID\footnote{%
    For example, the multicast group address for a service transfer with the destination node-ID of 42 is 239.1.0.42.
    Observe that multicast groups are not differentiated by service-ID.
}.

The IP address of a node bears no relevance for the protocol ---
multiple nodes may share the same IP address; likewise, a node may have more than one IP address.
Nodes on a Cyphal/UDP network are identified exclusively by their node-ID value\footnote{%
    A node that is registered on an IP network (e.g., via DHCP)
    still needs to obtain a node-ID value to participate in a Cyphal/UDP network.
    This may be done either through manual assignment or by using the plug-and-play node-ID allocation service
    (section~\ref{sec:application_functions_pnp}).
}.
The set of valid node-ID values for Cyphal/UDP is $[0, 65534]$.
Value 65535 is reserved to represent both the broadcast and anonymous node-ID, depending on context.

\begin{remark}
    Per RFC~1112, in order to emit multicast traffic,
    a limited level-1 implementation without the full support of IGMP and multicast-specific packet handling policies
    is sufficient.
    Thus, trivial nodes that are only required to publish messages on the network may be implemented
    without the need for a full IGMP stack.

    The reliance on IP multicasting exclusively allows baremetal implementations to omit ARP support.
\end{remark}

\begin{remark}
    Due to the dynamic nature of the IGMP protocol,
    a newly configured subscriber may not immediately receive data from the subject ---
    a brief subscription initialization delay may occur
    because the underlying IGMP stack needs to inform the router about its interest
    in the specified multicast group by sending an IGMP membership report first.
    Certain high-integrity applications may choose to rely on static switch configurations
    to eliminate the subscription initialization delay.
\end{remark}

\subsubsection{TTL}

Sources of Cyphal/UDP traffic should set the packet TTL to 16 or higher.

\begin{remark}
    RFC~1112 prescribes a default TTL of 1,
    but this is not sufficient as Cyphal/UDP networks may often exceed the diameter of 1 hop.
\end{remark}

\subsubsection{QoS}

The DSCP\footnote{RFC~2474} field of outgoing IP packets \emph{should}
be populated based on the Cyphal transfer priority level (section~\ref{sec:transport_transfer_priority})
as specified\footnote{%
    The recommended DSCP mapping is derived from RFC~8837 and RFC~3662.
    The implementation of suitable network policies is outside of the scope of this document.
    RFC~4594 provides a starting point for the design of such policies.
} in table \ref{table:transport_udp_priority}.
Implementations \emph{should} provide a means to override the default DSCP mapping per node\footnote{
    E.g., via configuration registers described in section~\ref{sec:application_functions_register}.
}.
All nodes in a Cyphal/UDP network \emph{shall} be configured with identical DSCP mapping
that is consistent with the QoS policies implemented in the network\footnote{%
    This requirement is intended to prevent inconsistent QoS treatment of Cyphal/UDP traffic and priority inversion.
}.

\begin{CyphalSimpleTable}{
    Recommended mapping from Cyphal priority level to DSCP\label{table:transport_udp_priority}
}{|l l l l|}
    Cyphal priority & Header priority value (sec. \ref{sec:transport_udp_payload}) & DSCP class & DSCP value \\
    Exceptional     & 0     & EF            & 46    \\
    Immediate       & 1     & EF            & 46    \\
    Fast            & 2     & EF            & 46    \\
    High            & 3     & DF            & 0     \\
    Nominal         & 4     & DF            & 0     \\
    Low             & 5     & DF            & 0     \\
    Slow            & 6     & DF            & 0     \\
    Optional        & 7     & CS1           & 8     \\
\end{CyphalSimpleTable}

\subsection{UDP datagram payload format}\label{sec:transport_udp_payload}

The layout of the Cyphal/UDP datagram payload header is shown in the following snippet in DSDL notation
(section~\ref{sec:dsdl}).
The payload header is followed by the payload data, which is opaque to the protocol.

\begin{samepage}
\begin{minted}{python}
# This 24-byte header can be aliased as a C structure with each field being naturally aligned:
#
#       uint8_t     version;
#       uint8_t     priority;
#       uint16_t    source_node_id;
#       uint16_t    destination_node_id;
#       uint16_t    data_specifier_snm;
#       uint64_t    transfer_id;
#       uint32_t    frame_index_eot;
#       uint16_t    user_data;
#       uint8_t[2]  header_crc16_big_endian;

uint4 version
# The version of the header format. This document specifies version 1.
# Packets with an unknown version number must be ignored.

void4

uint3 priority
# The values are assigned from 0 (HIGHEST priority) to 7 (LOWEST priority).
# The numerical priority identifiers are chosen to be consistent with Cyphal/CAN.
# The mapping from this priority value to the DSCP value should be configurable;
# otherwise, see the recommended default mapping.

void5

uint16 source_node_id
# The node-ID of the source node.
# Value 65535 represents anonymous transfers.

uint16 destination_node_id
# The node-ID of the destination node.
# Value 65535 represents broadcast transfers.

uint15 data_specifier
# If this is a message transfer, this value equals the subject-ID.
# If this is a service response transfer, this value equals the service-ID.
# If this is a service request transfer, this value equals 16384 + service-ID.

bool service_not_message
# If true, this is a service transfer. If false, this is a message transfer.

@assert _offset_ == {64}
uint64 transfer_id
# The monotonic transfer-ID value of the current transfer (never overflows).

uint31 frame_index
# Zero for a single-frame transfer and for the first frame of a multi-frame transfer.
# Incremented by one for each subsequent frame of a multi-frame transfer.

bool end_of_transfer
# True if this is the last frame of a multi-frame transfer, or a single-frame transfer.

uint16 user_data
# Opaque application-specific data with user-defined semantics.
# Generic implementations should emit zero and ignore this field upon reception.

uint8[2] header_crc16_big_endian
# CRC-16/CCITT-FALSE of the preceding serialized header data in the big endian byte order.
# Application of the CRC function to the entire header shall yield zero, otherwise the header is malformed.

@assert _offset_ / 8 == {24}
@sealed     # The payload data follows.
\end{minted}
\end{samepage}

The header CRC function is \textbf{CRC-16/CCITT-FALSE};
refer to section~\ref{sec:appendix_crc16ccitt_false} for further information.

\begin{remark}
    Certain states provided in the header duplicate information that is already available in the IP header
    or the multicast group address.
    This is done for reasons of unification of the header format with other standard transport layer definitions,
    and to simplify the access to the transfer parameters that otherwise would be hard to reach above the
    network layer, such as the DSCP value.
    The latter consideration is particularly important for forwarding nodes.
\end{remark}

\subsection{Transfer payload}

After the transfer payload is constructed but before it is scheduled for transmission over the network,
it is appended with the transfer CRC field.
The transfer CRC function is \textbf{CRC-32C} (section~\ref{sec:appendix_crc32c}),
and its value is serialized in the little-endian byte order.
The transfer CRC function is applied to the entire transfer payload and only transfer payload.

The transfer CRC is provided for all transfers,
including single-frame transfers and transfers with an empty payload\footnote{%
    This provides end-to-end integrity protection for the transfer payload.
}.
An implementation receiving a transfer should verify the correctness of its transfer CRC.

\begin{remark}
    From the perspective of the multi-frame segmentation logic, the transfer CRC field is part of the transfer payload.
    From the definition of the header format it follows that the transfer CRC can only be found at the end of
    the packet if the \verb|end_of_transfer| bit is set,
    unless the transfer CRC field has spilled over to the next frame
    (in which case the frame would contain only the transfer CRC itself or the tail thereof).
\end{remark}

\subsection{Maximum transmission unit}

In this section, the maximum transmission unit (MTU) is defined as the maximum size of a UDP/IP datagram payload.

This specification does not restrict the MTU of the underlying transport.
It is recommended, however, to avoid MTU values less than 508~bytes,
allowing applications to exchange up to $508 - 24 - 4 = 480$ bytes of payload in a single-frame transfer.
Limiting the MTU at this value allows nodes that do not transmit and/or receive transfers larger than 480~bytes
to omit support for multi-frame transfer decomposition and/or reassembly.

Nodes emitting Cyphal/UDP traffic \emph{should} avoid IP fragmentation by
limiting the size of the UDP payload accordingly\footnote{
    Transfers larger than the MTU limit will be transmitted as multi-frame Cyphal transfers
    (section~\ref{sec:transport_transfer_payload}).
    The preference towards Cyphal framing over IP framing is to remove the limitation on the maximum transfer size
    imposed by the UDP protocol and to permit preemption of long transfers by higher-priority transfers.
    Further, the support for reassembly of large fragmented IP packets is not universal.
    Therefore, constrained implementations may implement a compliant Cyphal/UDP stack without support for IP
    fragmentation.
}.

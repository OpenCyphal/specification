\chapter{Introduction}\label{sec:introduction}

UAVCAN is a lightweight protocol designed to provide a highly reliable communication method for
aerospace and robotic applications via robust vehicle bus networks.

This is a non-normative section covering the basic concepts that govern development and maintenance of
the specification.

\section{Design principles}

\begin{description}
    \item[Democratic network] --- There will be no master node.
    All nodes in the network will have the same communication rights; there must be no single point of failure.

    \item[High-level communication abstractions] --- The protocol must support publish-subscribe and remote procedure
    call semantics with statically defined and statically verified data types (schema).
    The data types used for communication must be defined in a clear, platform-agnostic way.

    \item[Facilitation of cross-vendor interoperability] --- UAVCAN should provide a common foundation that
    different vendors can use to ensure interoperability of their equipment.
    UAVCAN must provide a generic set of standard application-agnostic communication data types.

    \item[Well-defined generic high-level functions] --- UAVCAN must define standard services
    and messages for common high-level functions, such as network discovery, node configuration,
    node software update, node status monitoring (which naturally grows into a vehicle-wide health monitoring),
    network-wide time synchronization, plug-and-play node support, etc.

    \item[Atomic data abstractions] --- Nodes must be provided with a simple way of exchanging large
    data structures that exceed the capacity of a single transport frame\footnote{A \emph{transport frame} is
    an atomic transmission unit defined by the underlying transport protocol. For example, a CAN frame.}.
    UAVCAN should perform automatic data decomposition and reassembly at the protocol level,
    hiding the related complexity from the application.

    \item[High throughput, low latency communication] --- UAVCAN must be suitable for use in
    hard real-time systems.

    \item[Support for redundant interfaces and redundant nodes] --- UAVCAN must be suitable for use in
    applications that require modular redundancy.

    \item[Simple logic, low computational requirements] --- UAVCAN targets a wide variety of embedded systems,
    from high-performance embedded on-board computers for intensive data processing
    (e.g., a high-performance GNU/Linux-powered machine) to extremely resource-constrained microcontrollers.
    The latter imposes severe restrictions on the amount of logic needed to implement the protocol.

    \item[Support for various transport protocols] --- UAVCAN must be usable with different bus protocols,
    such as CAN 2.0 or CAN FD.
    The standard must be capable of accommodating other transport protocols in the future.

    \item[Open specification and reference implementations] --- The UAVCAN specification will always be open and
    free to use for everyone; the reference implementations will be distributed under the terms of
    the permissive MIT License or released into public domain.
\end{description}

\section{Capabilities}

UAVCAN-based networks can accommodate up to 127 nodes on the same logical bus.

UAVCAN supports an unlimited number of data types, which can be either standard (defined by the specification)
or vendor-specific (defined by third parties for private use or for public release).
There can be up to 256 major versions of a data type, and up to 256 minor versions per major version.
More information is provided in chapter \ref{sec:dsdl}.

UAVCAN supports up to 65536 message subject identifiers for publish/subscribe exchanges,
256 service identifiers for remote procedure call exchanges,
and at least 8 anonymous message subject identifiers for certain special features such as plug-and-play support.
A small subset of these identifiers is reserved for the core standard and for publicly released vendor-specific types
(data types with a predefined subject ID or service ID are said to have a \emph{regulated port ID}).
More information is provided in chapter \ref{sec:application_layer}.

UAVCAN supports at least\footnote{Depending on the transport protocol.} eight distinct communication priority levels,
defined in section \ref{sec:transfer_prioritization}.
Within each priority level different types of transfers and different data types are
prioritized in a well-defined, deterministic manner.

The list of transport protocols supported by UAVCAN is provided in chapter \ref{sec:transport_layer}.
Non-redundant, doubly-redundant and triply-redundant transports are supported.
More information on the physical layer and standardized physical connectivity options
is provided in chapter \ref{sec:physical_layer}.

\section{Maintenance of the standard data type set}

The UAVCAN maintainers are charged with advancing the standard data type set based on input from adopters.
This feedback is gathered via the official discussion
forum\footnote{Please refer to \href{http://uavcan.org}{uavcan.org}.}
which is open to the general public.

The set of standard data type definitions\footnote{The Data Structure Description Language (DSDL) and
related concepts are described in section \ref{sec:dsdl}.} is an integral part of the specification;
however, there is only a small set of required data types needed to implement the protocol.
A larger set of optional data types are defined to create a standardized data exchange environment
supporting the interoperability of COTS\footnote{Commercial off-the-shelf equipment.}
equipment manufactured by different vendors.
See chapter \ref{sec:application_layer} for more information.

Within the same major version of the specification,
the set of standard data type definitions can be modified only in the following ways:

\begin{itemize}
    \item A new data type can be added, as long as it doesn't conflict with any of the existing data types.

    \item An existing data type can be modified, as long as its version number is updated accordingly
    and all backward compatibility guarantees are respected.

    \item An existing data type or a particular major version of it can be declared deprecated.
    \begin{itemize}
        \item Once declared deprecated, the data type will be maintained for at least two more years.
        After this period, its regulated port ID (if defined) may be reused for an incompatible data type definition.
        \item Deprecation will be indicated in the DSDL definition and announced via the discussion forum.
    \end{itemize}
\end{itemize}

A link to the repository containing the set of default DSDL definitions can be found on the official
website\footnote{\href{http://uavcan.org}{uavcan.org}}.

\section{Referenced sources}

The UAVCAN specification contains references to the following sources:

\begin{itemize}
    \item CiA 801 --- Application note --- Automatic bit rate detection.
    \item CiA 103 --- Intrinsically safe capable physical layer.
    \item CiA 303 --- Recommendation --- Part 1: Cabling and connector pin assignment.
    \item IEEE 754 --- Standard for binary floating-point arithmetic.
    \item ISO 11898-1 --- Controller area network (CAN) --- Part 1: Data link layer and physical signaling.
    \item ISO 11898-2 --- Controller area network (CAN) --- Part 2: High-speed medium access unit.
    \item ISO/IEC 10646 --- Universal Coded Character Set (UCS).
    \item ISO/IEC 14882 --- Programming Language C++.
    \item ``Implementing a Distributed High-Resolution Real-Time Clock using the CAN-Bus'', M. Gergeleit and H. Streich.
    \item ``In Search of an Understandable Consensus Algorithm (Extended Version)'', Diego Ongaro and John Ousterhout.
    \item \href{http://semver.org}{semver.org} - Semantic versioning specification.
\end{itemize}

\chapter{Introduction}\label{sec:introduction}

This is a non-normative chapter covering the basic concepts that govern development and maintenance of
the specification.

\section{Overview}

Cyphal is a lightweight protocol designed to provide a highly reliable communication method
supporting publish-subscribe and remote procedure call semantics
for aerospace and robotic applications via robust vehicle bus networks.
It is created to address the challenge of deterministic on-board data exchange between systems and components
of next-generation intelligent vehicles: manned and unmanned aircraft, spacecraft, robots, and cars.

Cyphal can be approximated as a highly deterministic decentralized object request broker
with a specialized interface description language and a highly efficient data serialization format
suitable for use in real-time safety-critical systems with optional modular redundancy.

The name Cyphal stands for \emph{Uncomplicated Application-level Vehicular Computing And Networking}.

Cyphal is a standard open to everyone, and it will always remain this way.
No authorization or approval of any kind is necessary for its implementation, distribution, or use.

The development and maintenance of the Cyphal specification is governed through the public discussion forum,
software repositories, and other resources available via the official website at
\href{http://opencyphal.org}{opencyphal.org}.

Engineers seeking to leverage Cyphal should also consult with \emph{The Cyphal Guide} --
a separate textbook available via the official website.

\section{Document conventions}

Non-normative text, examples, recommendations, and elaborations that do not directly participate
in the definition of the protocol are contained in footnotes\footnote{This is a footnote.}
or highlighted sections as shown below.

\begin{remark}
    Non-normative sections such as examples are enclosed in shaded boxes like this.
\end{remark}

Code listings are formatted as shown below.
All such code is distributed under the same license as this specification, unless specifically stated otherwise.

\begin{minted}{rust}
    // This is a source code listing.
    fn main() {
        println!("Hello World!");
    }
\end{minted}

A byte is a group of eight (8) bits.

Textual patterns are specified using the standard
POSIX Extended Regular Expression (ERE) syntax;
the character set is ASCII and patterns are case sensitive, unless explicitly specified otherwise.

Type parameterization expressions use subscript notation,
where the parameter is specified in the subscript enclosed in angle brackets:
$\texttt{type}_\texttt{<parameter>}$.

Numbers are represented in base-10 by default.
If a different base is used, it is specified after the number in the subscript\footnote{%
    E.g., $\text{BADC0FFEE}_{16} = 50159747054$, $10101_2 = 21$.
}.

DSDL definition examples provided in the document are illustrative and may be incomplete or invalid.
This is to ensure that the examples are not cluttered by irrelevant details.
For example, \verb|@extent| or \verb|@sealed| directives may be omitted if not relevant.

\section{Design principles}

\begin{description}
    \item[Democratic network] --- There will be no master node.
    All nodes in the network will have the same communication rights; there should be no single point of failure.

    \item[Facilitation of functional safety] --- A system designer relying on Cyphal will have the necessary
    guarantees and tools at their disposal to analyze the system and ensure its correct behavior.

    \item[High-level communication abstractions] --- The protocol will support publish/subscribe and remote procedure
    call communication semantics with statically defined and statically verified data types (schema).
    The data types used for communication will be defined in a clear, platform-agnostic way
    that can be easily understood by machines, including humans.  % I hope you are ok with this, my dear fellow robots.

    \item[Facilitation of cross-vendor interoperability] --- Cyphal will be a common foundation that
    different vendors can build upon to maximize interoperability of their equipment.
    Cyphal will provide a generic set of standard application-agnostic communication data types.

    \item[Well-defined generic high-level functions] --- Cyphal will define standard services
    and messages for common high-level functions, such as network discovery, node configuration,
    node software update, node status monitoring, network-wide time synchronization, plug-and-play node support, etc.

    \item[Atomic data abstractions] --- Nodes shall be provided with a simple way of exchanging large
    data structures that exceed the capacity of a single transport frame\footnote{%
        A \emph{transport frame} is an atomic transmission unit defined by the underlying transport protocol.
        For example, a CAN frame.
    }.
    Cyphal should perform automatic data decomposition and reassembly at the protocol level,
    hiding the related complexity from the application.

    \item[High throughput, low latency, determinism] --- Cyphal will add a very low overhead to the underlying
    transport protocol, which will ensure high throughput and low latency, rendering the protocol well-suited
    for hard real-time applications.

    \item[Support for redundant interfaces and redundant nodes] --- Cyphal shall be suitable for use in
    applications that require modular redundancy.

    \item[Simple logic, low computational requirements] --- Cyphal targets a wide variety of embedded systems,
    from high-performance on-board computers to extremely resource-constrained microcontrollers.
    It will be inexpensive to support in terms of computing power and engineering hours,
    and advanced features can be implemented incrementally as needed.

    \item[Rich data type and interface abstractions] --- An interface description language will be a core part of
    the technology which will allow deeply embedded sub-systems to interface with higher-level systems directly and
    in a maintainable manner while enabling simulation and functional testing.

    \item[Support for various transport protocols] --- Cyphal will be usable with different transports.
    The standard shall be capable of accommodating other transport protocols in the future.

    \item[API-agnostic standard] --- Unlike some other networking standards, Cyphal will not attempt to describe
    the application program interface (API). Any details that do not affect the behavior of an implementation
    observable by other participants of the network will be outside of the scope of this specification.

    \item[Open specification and reference implementations] --- The Cyphal specification will always be open and
    free to use for everyone; the reference implementations will be distributed under the terms of
    the permissive MIT License or released into the public domain.
\end{description}

\section{Capabilities}

The maximum number of nodes per logical network is dependent on the transport protocol in use,
but it is guaranteed to be not less than 128.

Cyphal supports an unlimited number of composite data types,
which can be defined by the specification (such definitions are called \emph{standard data types})
or by others for private use or for public release
(in which case they are said to be \emph{application-specific} or \emph{vendor-specific}; these terms are equivalent).
There can be up to 256 major versions of a data type, and up to 256 minor versions per major version.

Cyphal supports 8192 message subject identifiers for publish/subscribe exchanges and
512 service identifiers for remote procedure call exchanges.
A small subset of these identifiers is reserved for the core standard and for publicly released vendor-specific types
(chapter~\ref{sec:application}).

Depending on the transport protocol, Cyphal supports at least eight distinct communication priority levels
(section~\ref{sec:transport_transfer_priority}).

The list of transport protocols supported by Cyphal is provided in chapter~\ref{sec:transport}.
Non-redundant, doubly-redundant and triply-redundant transports are supported.
Additional transport layers may be added in future revisions of the protocol.

Application-level capabilities of the protocol (such as time synchronization, file transfer,
node software update, diagnostics, schemaless named registers, diagnostics, plug-and-play node insertion, etc.)
are listed in section~\ref{sec:application_functions}.

The core specification does not define nor explicitly limit any physical layers for a given transport, however;
properties required by Cyphal may imply or impose constraints and/or minimum performance requirements on physical
networks. Because of this, the core standard does not control compatibility below a supported transport layer between
compliant nodes on a physical network (i.e. there are no, anticipated, compatibility concerns between compliant nodes
connected to a virtual network where hardware constraints are not enforced nor emulated).
Additional standards specifying physical-layer requirements, including connectors,
may be required to utilize this standard in a vehicle system.

The capabilities of the protocol will never be reduced within a major version of the specification but may be expanded.

\section{Management policy}

The Cyphal maintainers are tasked with maintaining and advancing this specification and
the set of public regulated data types\footnote{%
    The related technical aspects are covered in chapters~\ref{sec:basic} and~\ref{sec:dsdl}.
} based on their research and the input from adopters.
The maintainers will be committed to ensuring long-term stability and backward compatibility of
existing and new deployments.
The maintainers will publish relevant announcements and solicit inputs from adopters
via the discussion forum whenever a decision that may potentially affect existing deployments is being made.

The set of standard data types is a subset of public regulated data types and is an integral part of the specification;
however, there is only a very small subset of required standard data types needed to implement the protocol.
A larger set of optional data types are defined to create a standardized data exchange environment
supporting the interoperability of COTS\footnote{Commercial off-the-shelf equipment.}
equipment manufactured by different vendors.
Adopters are invited to take part in the advancement and maintenance of the public regulated data types
under the management and coordination of the Cyphal maintainers.

\section{Referenced sources}

The Cyphal specification contains references to the following sources:

% Please keep the list sorted alphabetically.
\begin{itemize}
    \item CiA 103 --- Intrinsically safe capable physical layer.
    \item CiA 801 --- Application note --- Automatic bit rate detection.

    \item IEEE 754 --- Standard for binary floating-point arithmetic.
    \item IEEE Std 1003.1 --- IEEE Standard for Information Technology --
          Portable Operating System Interface (POSIX) Base Specifications.

    \item IETF RFC2119 --- Key words for use in RFCs to Indicate Requirement Levels.

    \item ISO 11898-1 --- Controller area network (CAN) --- Part 1: Data link layer and physical signaling.
    \item ISO 11898-2 --- Controller area network (CAN) --- Part 2: High-speed medium access unit.
    \item ISO/IEC 10646 --- Universal Coded Character Set (UCS).
    \item ISO/IEC 14882 --- Programming Language C++.

    \item \href{http://semver.org}{semver.org} --- Semantic versioning specification.

    \item ``A Passive Solution to the Sensor Synchronization Problem'', Edwin Olson.
    \item ``Implementing a Distributed High-Resolution Real-Time Clock using the CAN-Bus'', M. Gergeleit and H. Streich.
    \item ``In Search of an Understandable Consensus Algorithm (Extended Version)'', Diego Ongaro and John Ousterhout.
\end{itemize}

\section{Revision history}

\subsection{v1.0 -- work in progress}

\begin{itemize}
    \item The maximum data type name length has been increased from 50 to 255 characters.

    \item The default extent function has been removed (section \ref{sec:dsdl_composite_extent_and_sealing}).
    The extent now has to be specified explicitly always unless the data type is sealed.
\end{itemize}

\subsection{v1.0-beta -- Sep 2020}

Compared to v1.0-alpha, the differences are as follows (the motivation is provided on the forum):

\begin{itemize}
    \item The physical layer specification has been removed.
    It is now up to the domain-specific Cyphal-based standards to define the physical layer.

    \item The subject-ID range reduced from $[0, 32767]$ down to $[0, 8191]$.
    This change may be reverted in a future edition of the standard, if found practical.

    \item Added support for delimited serialization; introduced related concepts of \emph{extent} and \emph{sealing}
    (section \ref{sec:dsdl_composite_extent_and_sealing}).
    This change enables one to easily evolve networked services in a backward-compatible way.

    \item Enabled the automatic runtime adjustment of the transfer-ID timeout on a per-subject basis
    as a function of the transfer reception rate (section \ref{sec:transport_transfer_reception}).
\end{itemize}

\subsection{v1.0-alpha -- Jan 2020}

This is the initial version of the document.
The discussions that shaped the initial version are available on the public Cyphal discussion forum.

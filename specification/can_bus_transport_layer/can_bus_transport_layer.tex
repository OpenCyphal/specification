\chapter{CAN bus transport layer}\label{sec:can_bus_transport_layer}

This chapter defines the CAN bus based transport layer of UAVCAN,
both for the legacy CAN 2.0 and for CAN FD.
The differences between the two transports are minor,
so by default all of the information provided in this chapter applies to both,
unless specifically stated otherwise.

\section{The concept of transfer}

A \emph{transfer} is an act of data transmission between nodes.
A transfer that is addressed to all nodes except the source node is a \emph{broadcast transfer}.
A transfer that is addressed to one particular node is a \emph{unicast transfer}.
UAVCAN defines the following types of transfers:

\begin{description}
    \item[Message transfer] - a broadcast transfer that contains a serialized message.
    \item[Service transfer] - a unicast transfer that contains either a service request or a service response.
\end{description}

Both message and service transfers can be further distinguished between:

\begin{description}
    \item[Single-frame transfer] - a transfer that is entirely contained in a single CAN frame.
    \item[Multi-frame transfer] - a transfer that has its payload distributed over multiple CAN frames.
    The UAVCAN protocol stack handles transfer decomposition and reassembly automatically.
\end{description}

The following properties are common to all types of transfers:

\begin{UAVCANSimpleTable}{Common transfer properties}{|l X|}\label{table:common_transfer_properties}
    Property        & Description \\
    Payload         & The serialized data structure. \\
    Data type ID    & A numerical identifier that indicates how the data structure should be interpreted. \\
    Data type major version number & Semantic major version number of the data type definition. \\
    Source node ID  & The node ID of the transmitting node (excepting anonymous message transfers). \\
    Priority        & A non-negative integer value that defines the message urgency (0 is the highest priority).
                      Higher priority transfers can preempt lower priority transfers. \\
    Transfer ID     & A small overflowing integer that increments with every transfer
                      of this type of message from a given node. \\
\end{UAVCANSimpleTable}

\subsection{Message broadcasting}

\subsubsection{Regular message broadcasting}

Message broadcasting is the main method of communication between UAVCAN nodes.

A broadcast message is carried by a single message transfer that contains the serialized message data structure.
A broadcast message does not contain any additional fields besides those listed in the table
\ref{table:common_transfer_properties}.

In order to broadcast a message, the broadcasting node must have a node ID that is unique within the network.
An exception applies to \emph{anonymous message broadcasts}.

\subsubsection{Anonymous message broadcasting}

An anonymous message transfer is a transfer that originates from a node that does not have a node ID.
This sort of message transfer is especially useful for \emph{dynamic node ID allocation}
(a high-level concept that is reviewed in detail in the chapter \ref{sec:application_layer}).

A node that does not have a node ID is said to be in \emph{passive mode}.
Passive nodes are unable to initiate regular data exchanges,
but they can listen to the data exchanged over the bus,
and they can emit anonymous message transfers.

An anonymous message has the same properties as a regular message, except for the source node ID,
which in the case of anonymous message transfers is always assumed to be zero.

An anonymous transfer can only be a single-frame transfer. Multi-frame anonymous message transfers are not allowed.

Note that anonymous messages require specific arbitration rules and have restrictions on the acceptable
data type ID values. The details are explained later in this chapter.

\subsubsection{Message timing requirements}

Generally, a message transmission should be aborted if it cannot be completed in 1 second.
Applications are allowed to deviate from this recommendation,
provided that every such deviation is explicitly documented.
It is expected that high-frequency high-priority messages may opt for lower timeout values,
whereas low-priority data may opt for higher timeout values to account for CAN bus congestion.

\subsection{Service invocation}

A service invocation sequence consists of two related service transfers:

\begin{description}
    \item[Service request transfer] - from the node that invokes the service - the \emph{client} - to the node that
    provides the service - the \emph{server}.

    \item[Service response transfer] - once the \emph{server node} receives the service request and processes it,
    it sends a response transfer back to the \emph{client node}.
\end{description}

The tables \ref{table:service_request_transfer_properties} and \ref{table:service_response_transfer_properties}
describe the properties of service request and service response transfers, respectively.

Both the client and the server must have node ID values that are unique within the network;
service invocation is not available to passive nodes.

\begin{UAVCANSimpleTable}{Service request transfer properties}{|l X|}\label{table:service_request_transfer_properties}
    Property                        & Description \\
    Payload                         & The serialized service request data structure. \\
    Data type ID                    & See the table \ref{table:common_transfer_properties}. \\
    Data type major version number  & See the table \ref{table:common_transfer_properties}. \\
    Source node ID                  & The node ID of the client (the invoking node). \\
    Destination node ID             & The node ID of the server (the invoked node). \\
    Priority                        & See the table \ref{table:common_transfer_properties}. \\
    Transfer ID                     & An integer value that:
        \begin{enumerate}
            \item allows the server to distinguish the request from other requests from the same client;
            \item allows the client to match the response with its request.
        \end{enumerate} \\
\end{UAVCANSimpleTable}

\begin{UAVCANSimpleTable}{Service response transfer properties}{|l X|}\label{table:service_response_transfer_properties}
    Property                        & Description \\
    Payload                         & The serialized service response data structure. \\
    Data type ID                    & Same value as in the request transfer. \\
    Data type major version number  & Same value as in the request transfer. \\
    Source node ID                  & The node ID of the server (the invoked node). \\
    Destination node ID             & The node ID of the client (the invoking node). \\
    Priority                        & Same value as in the request transfer. \\
    Transfer ID                     & Same value as in the request transfer. \\
\end{UAVCANSimpleTable}

\subsubsection{Service timing requirements}

Applications should follow the service invocation timing recommendations specified below.
Applications are allowed to deviate from these recommendations,
provided that every such deviation is explicitly documented.

\begin{itemize}
    \item Service transfer transmission should be aborted if does not complete in 1 second.
    \item The client should stop waiting for a response from the server if one has not arrived within 1 second.
    \item The server should be able to process any request in under 0.5 seconds.
\end{itemize}

\subsection{Transfer prioritization}\label{sec:transfer_prioritization}

UAVCAN transfers are prioritized by means of the transfer priority property,
which allows eight different priority levels for all types of transfers.
The priority levels and the corresponding numerical identifiers are specified
in the table \ref{table:transfer_priority_levels}.
Observe that due to the specifics of the CAN bus,
lower numerical values correspond to higher priority levels.
The human-friendly mnemonics are introduced in order to prevent confusion due to the inverted nature of the
priority level identifiers.

\begin{minipage}{0.6\textwidth}
\begin{UAVCANSimpleTable}{Transfer priority levels}{|l X|}\label{table:transfer_priority_levels}
    Numerical ID            & Mnemonic \\
    0 (highest priority)    & Emergency \\
    1                       & Critical \\
    2                       & Urgent \\
    3                       & High \\
    4                       & Normal \\
    5                       & Low \\
    6                       & Diagnostic \\
    7 (lowest priority)     & Background \\
\end{UAVCANSimpleTable}
\end{minipage}

Transfers with higher priority levels (i.e., numerically lower priority ID)
preempt transfers with lower priority levels, delaying their transmission
until there are no more higher priority transfers to exchange.

Shall there be multiple transfers of different types at the same priority level contesting for the bus access,
UAVCAN ensures the following precedence, from higher priority to lower priority:

\begin{enumerate}
    \item Message transfers.
    \item Service response transfers.
    \item Service request transfers.
\end{enumerate}

Message transfers take precedence over service transfers because message broadcasting is the primary method of
communication in UAVCAN networks.
Service responses take precedence over service requests in order to make service invocations more atomic
and reduce the number of pending states in the system.

Within the same type and the same priority level,
transfers are prioritized according to the data type ID:
transfers with lower data type ID values preempt those with higher data type ID values.

\section{Transfer emission}

\subsection{Transfer ID computation}




\section{CAN frame format}

\section{Transfer reception}

\section{CAN bus requirements}




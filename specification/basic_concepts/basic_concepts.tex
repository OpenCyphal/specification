\chapter{Basic concepts}\label{sec:basic_concepts}

UAVCAN is a lightweight protocol designed to provide a highly reliable communication method for
aerospace and robotic applications via robust vehicle bus networks.
A UAVCAN network is a decentralized peer network, where each peer (node) has a unique
numeric identifier --- \emph{node ID} --- ranging from 1 to 127, inclusively.
Nodes of a UAVCAN network can communicate using the following communication methods:
\begin{description}
    \item[Message broadcasting] --- The primary method of data exchange with one-to-all publish/subscribe semantics.
    \item[Service invocation] --- The communication method for peer-to-peer request/response
    interactions\footnote{Like remote procedure call (RPC).}.
\end{description}

For each type of communication, a predefined set of data structures is used,
where each data structure has a unique identifier --- the \emph{data type ID} (DTID).
Additionally, every data structure definition has a pair of major and minor semantic version numbers,
which enable data type definitions to evolve in arbitrary ways while ensuring a comprehensible
migration path in the event of backward-incompatible changes.
Some data structures are standard and defined by the protocol specification;
others may be specific to a particular application or vendor.

Since every message or service data type has its own unique data type ID,
and each node in the network has its own unique node ID,
a pair of data type ID and node ID can be used to support redundant nodes with identical
functionality inside the same network.

Message and service data structures are defined using the Data Structure Description Language
(DSDL) (chapter \ref{sec:dsdl}).
A DSDL description can be used to automatically generate the serialization/deserialization code
for every defined data structure in a particular programming language.
DSDL ensures that the worst case memory footprint and computational complexity per data type
are constant and easily predictable, which is paramount for hard real-time and safety-critical applications.

On top of the standard data types, UAVCAN defines a set of standard high-level functions including:
node health monitoring, network discovery, time synchronization, firmware update,
plug-and-play node support, and more.
For more information see chapter \ref{sec:application_layer}.

Serialized message and service data structures are exchanged by means of the transport
layer (chapter \ref{sec:transport_layer}),
which implements automatic decomposition/reassembly of long transfers into/from several transport
frames\footnote{Here and elsewhere, a \emph{transport frame} means a block of data
that can be atomically exchanged over the network, e.g., a CAN frame.},
allowing nodes to exchange data structures of arbitrary size.

\begin{figure}[hbt]
    \centering
    \setlength\arrayrulewidth{1pt}  % https://tex.stackexchange.com/a/256732/132781
    \begin{tabularx}{0.8\textwidth}{|X|X|X|X|}
        \hline
        \multicolumn{4}{|c|}{Application} \\
        \hline

        & Standard functions & \multicolumn{2}{c|}{\multirow{3}{*}{Vendor-specific data types (optional)}} \\\cline{2-2}
        \multicolumn{2}{|c|}{\multirow{2}{*}{Standard data types}} & \multicolumn{2}{c|}{} \\
        \multicolumn{2}{|c|}{} & \multicolumn{2}{c|}{} \\

        \hline
        \multicolumn{4}{|c|}{Serialization} \\
        \hline
        \multicolumn{4}{|c|}{Transport layer} \\
        \hline
    \end{tabularx}
    \caption{UAVCAN architectural diagram.\label{fig:architecture}}
\end{figure}

\section{Message broadcasting}

Message broadcasting refers to the transmission of a serialized data structure over the network to other nodes.
This is the primary data exchange mechanism used in UAVCAN;
it is functionally similar to raw data exchange with minimal overhead,
additional communication integrity guarantees, and automatic decomposition and reassembly of long payloads
across multiple transport frames.
Typical use cases may include transfer of the following kinds of data (either cyclically or on an ad-hoc basis):
sensor measurements, actuator commands, equipment status information, and more.

Information contained in a broadcast message is summarized in the table \ref{table:broadcast_message_info}.

\begin{UAVCANSimpleTable}{Broadcast message properties}{|l X|}\label{table:broadcast_message_info}
    Property        & Description \\
    Payload         & The serialized message data structure. \\
    Data type ID    & Numerical identifier that indicates how the data structure should be interpreted. \\
    Data type major version number & Semantic major version number of the data type description. \\
    Source node ID  & The node ID of the transmitting node (excepting anonymous messages). \\
    Transfer ID     & A small overflowing integer that increments with every transfer
                      of this message type from a given node. Used for message sequence monitoring,
                      multi-frame transfer reassembly, and elimination of transport frame duplication errors
                      for single-frame transfers. Additionally, Transfer ID is crucial for automatic
                      management of redundant transport interfaces. The properties of this field are explained in
                      detail in the chapter \ref{sec:transport_layer}. \\
\end{UAVCANSimpleTable}

\subsection{Anonymous message broadcasting}

Nodes that don't have a unique node ID can publish \emph{anonymous messages}.
An anonymous message is different from a regular message in that it doesn't contain a source node ID.

UAVCAN nodes will not have an identifier initially until they are assigned one,
either statically (which is generally the preferred option for applications where a high degree of
determinism and high safety assurances are required) or dynamically.
Anonymous messages are particularly useful for the dynamic node ID allocation feature,
which is explored in detail in the chapter~\ref{sec:application_layer}.

Anonymous messages cannot be decomposed into multiple transport frames,
meaning that their payload capacity is limited to that of a single transport frame.
More info is provided in the chapter~\ref{sec:transport_layer}.

\section{Service invocation}

Service invocation is a two-step data exchange operation between exactly two nodes: a client and a server.
The steps are\footnote{The request/response semantic is facilitated by means of hardware (if available)
or software acceptance filtering and higher-layer logic.
No additional support or non-standard transport layer features are required.}:

\begin{enumerate}
    \item The client sends a service request to the server.
    \item The server takes appropriate actions and sends a response to the client.
\end{enumerate}

Typical use cases for this type of communication include:
node configuration parameter update, firmware update, an ad-hoc action request, file transfer,
and similar service tasks.

Information contained in service requests and responses is summarized in the
table \ref{table:service_req_resp_info}.

\begin{UAVCANSimpleTable}{Service request/response properties}{|l X|}\label{table:service_req_resp_info}
    Property        & Description \\
    Payload         & The serialized request/response data structure. \\
    Data type ID    & Numerical identifier that indicates how the data structure should be interpreted. \\
    Data type major version number & Semantic major version number of the data type definition. \\
    Client node ID  & Source node ID during request transfer, destination node ID during response transfer. \\
    Server node ID  & Destination node ID during request transfer, source node ID during response transfer. \\
    Transfer ID     & A small overflowing integer that increments with every call
                      of this service type from a given node. Used for request/response matching,
                      multi-frame transfer reassembly, and elimination of transport frame duplication errors
                      for single-frame transfers. Additionally, Transfer ID is crucial for automatic
                      management of redundant transport interfaces. The properties of this field are explained in
                      detail in the chapter \ref{sec:transport_layer}. \\
\end{UAVCANSimpleTable}

Both request and response contain same values for all listed fields except payload,
where the content is application-defined.
Clients match responses with corresponding requests using the following fields:
data type ID, data type major version number, client node ID, server node ID, and transfer ID.

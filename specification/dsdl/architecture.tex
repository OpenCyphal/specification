\section{Architecture}

\subsection{General principles}

In accordance with the Cyphal architecture, DSDL allows users to define data types of two kinds:
message types and service types.
Message types are used to exchange data over publish-subscribe one-to-many message links identified by subject-ID,
and service types are used to perform request-response one-to-one exchanges (like RPC) identified by service-ID.
A service type is composed of exactly two inner data types:
one of them is the request type (its instances are transferred from client to server),
and the other is the response type (its instances are transferred from the server back to the client).

Following the deterministic nature of Cyphal, the size of a serialized representation of any
message or service object is bounded within statically known limits.
Variable-size entities always have a fixed size limit defined by the data type designer.

DSDL definitions are strongly statically typed.

DSDL provides a well-defined means of data type versioning, which enables data type maintainers to introduce changes
to released data types while ensuring backward compatibility with fielded systems.

DSDL is designed to support extensive static analysis. Important properties of data type definitions such as
backward binary compatibility and data field layouts can be checked and validated by automatic software tools
before the systems utilizing them are fielded.

DSDL definitions can be used to automatically generate serialization (and deserialization) source code
for any data type in a target programming language.
A tool that is capable of generating serialization code based on a DSDL definition is called a \emph{DSDL compiler}.
More generically, a software tool designed for working with DSDL definitions is called a
\emph{DSDL processing tool}.

\subsection{Data types and namespaces}

Every data type is located inside a \emph{namespace}.
Namespaces may be included into higher-level namespaces, forming a tree hierarchy.

A namespace that is at the root of the tree hierarchy (i.e., not nested within another one)
is called a \emph{root namespace}.
A namespace that is located inside another namespace is called a \emph{nested namespace}.

A data type is uniquely identified by its namespaces and its \emph{short name}.
The short name of a data type is the name of the type itself excluding the containing namespaces.

A \emph{full name} of a data type consists of its short name and all of its namespace names.
The short name and the namespace names included in a full name are called \emph{name components}.
Name components are ordered: the root namespace is always the first component of the name,
followed by the nested namespaces, if there are any, in the order of their nesting;
the short name is always the last component of the full name.
The full name is formed by joining its name components via the ASCII dot character ``\verb|.|'' (ASCII code 46).

A \emph{full namespace} name is the full name without the short name and its component separator.

A \emph{sub-root namespace} is a nested namespace that is located immediately under its root namespace.
Data types that reside directly under their root namespace do not have a sub-root namespace.

The name structure is illustrated in figure~\ref{fig:dsdl_data_type_name_structure}.

\begin{figure}[H]
    $$
    \overbrace{
        \underbrace{
            \underbrace{\texttt{\huge{uavcan}}}_{\substack{\text{root} \\ \text{namespace}}}%
            \texttt{\huge{.}}%
            \underbrace{\texttt{\huge{node}}}_{\substack{\text{nested, also} \\ \text{sub-root} \\ \text{namespace}}}%
            \texttt{\huge{.}}%
            \underbrace{\texttt{\huge{port}}}_{\substack{\text{nested} \\ \text{namespace}}}%
        }_{\text{full namespace}}%
        \texttt{\huge{.}}%
        \underbrace{\texttt{\huge{GetInfo}}}_{\text{short name}}
    }^{\text{full name}}
    $$
    \caption{Data type name structure\label{fig:dsdl_data_type_name_structure}}
\end{figure}

A set of full namespace names and a set of full data type names shall not intersect\footnote{%
    For example, a namespace ``\texttt{vendor.example}'' and a data type ``\texttt{vendor.example.1.0}''
    are mutually exclusive.
    Note the data type name shown in this example violates the naming conventions
    which are reviewed in a separate section.
}.

Data type names and namespace names are case-sensitive.
However, names that differ only in letter case are not permitted\footnote{%
    Because that may cause problems with case-insensitive file systems.
}.
In other words, a pair of names which differ only in letter case is considered to constitute a name collision.

A name component consists of alphanumeric ASCII characters (which are: \verb|A-Z|, \verb|a-z|, and \verb|0-9|)
and underscore (``\verb|_|'', ASCII code 95).
An empty string is not a valid name component.
The first character of a name component shall not be a digit.
A name component shall not match any of the reserved word patterns,
which are listed in table~\ref{table:dsdl_reserved_word_patterns}.

The length of a full data type name shall not exceed 255
characters\footnote{This includes the name component separators, but not the version.}.

Every data type definition is assigned a major and minor version number pair.
In order to uniquely identify a data type definition, its version numbers shall be specified.
In the following text, the term \emph{version} without a majority qualifier refers to
a pair of major and minor version numbers.

Valid data type version numbers range from 0 to 255, inclusively.
A data type version where both major and minor components are zero is not allowed.

\subsection{File hierarchy}

DSDL data type definitions are contained in UTF-8 encoded text files with a file name extension \verb|.dsdl|.

One file defines exactly one version of a data type,
meaning that each combination of major and minor version numbers shall be unique per data type name.
There may be an arbitrary number of versions of the same data type defined alongside each other,
provided that each version is defined at most once.
Version number sequences can be non-contiguous,
meaning that it is allowed to skip version numbers or remove existing definitions that are neither oldest nor newest.

A data type definition may have an optional fixed port-ID\footnote{Chapter~\ref{sec:basic}.} value specified.

The name of a data type definition file is constructed from the following entities
joined via the ASCII dot character ``\verb|.|'' (ASCII code 46), in the specified order:
\begin{itemize}
    \item Fixed port-ID in decimal notation, unless a fixed port-ID is not provided for this definition.
    \item Short name of the data type (mandatory, always non-empty).
    \item Major version number in decimal notation (mandatory).
    \item Minor version number in decimal notation (mandatory).
    \item File name extension ``\verb|dsdl|'' (mandatory).
\end{itemize}

\begin{figure}[H]
    $$
    \overbrace{%
        \underbrace{\texttt{\huge{432}}}_{\substack{\text{fixed} \\ \text{port-ID}}}%
        \texttt{\huge{.}}%
    }^{\text{optional}}%
    \overbrace{%
        \underbrace{\texttt{\huge{GetInfo}}}_{\substack{\text{short name}}}%
        \texttt{\huge{.}}%
        \underbrace{\texttt{\huge{1.0}}}_{\substack{\text{version} \\ \text{numbers}}}%
        \texttt{\huge{.}}%
        \underbrace{\texttt{\huge{dsdl}}}_{\text{file extension}}%
    }^{\text{mandatory}}
    $$
    \caption{Data type definition file name structure\label{fig:dsdl_definition_file_name_structure}}
\end{figure}

DSDL namespaces are represented as directories, where one directory defines exactly one namespace, possibly nested.
The name of the directory defines the name of its data type name component.
A directory defining a namespace will always define said namespace in its entirety,
meaning that the contents of a namespace cannot be spread across different directories sharing the same name.
One directory cannot define more than one level of
nesting\footnote{%
    For example, ``\texttt{foo.bar}'' is not a valid directory name.
    The valid representation would be ``\texttt{bar}'' nested in ``\texttt{foo}''.
}.

\begin{remark}
    \begin{figure}[H]
        \begin{tabu}{|l|X|} \hline
            \rowfont{\bfseries}
            Directory tree & Entry description \\\hline

            \texttt{vendor\_x/} &
            Root namespace \texttt{vendor\_x}. \\\cline{2-2}

            \texttt{\qquad{}foo/} &
            Nested namespace (also sub-root) \texttt{vendor\_x.foo}. \\\cline{2-2}

            \texttt{\qquad{}\qquad{}100.Run.1.0.dsdl} &
            Data type definition v1.0 with fixed service-ID 100. \\\cline{2-2}

            \texttt{\qquad{}\qquad{}100.Status.1.0.dsdl} &
            Data type definition v1.0 with fixed subject-ID 100. \\\cline{2-2}

            \texttt{\qquad{}\qquad{}ID.1.0.dsdl} &
            Data type definition v1.0 without fixed port-ID. \\\cline{2-2}

            \texttt{\qquad{}\qquad{}ID.1.1.dsdl} &
            Data type definition v1.1 without fixed port-ID. \\\cline{2-2}

            \texttt{\qquad{}\qquad{}bar\_42/} &
            Nested namespace \texttt{vendor\_x.foo.bar\_42}. \\\cline{2-2}

            \texttt{\qquad{}\qquad{}\qquad{}101.List.1.0.dsdl} &
            Data type definition v1.0 with fixed service-ID 101. \\\cline{2-2}

            \texttt{\qquad{}\qquad{}\qquad{}102.List.2.0.dsdl} &
            Data type definition v2.0 with fixed service-ID 102. \\\cline{2-2}

            \texttt{\qquad{}\qquad{}\qquad{}ID.1.0.dsdl} &
            Data type definition v1.0 without fixed port-ID. \\\hline
        \end{tabu}
        \caption{DSDL directory structure example}\label{fig:dsdl_directory_structure_example}
    \end{figure}
\end{remark}

\subsection{Elements of data type definition}\label{sec:dsdl_elements_of_data_type_definition}

A data type definition file contains an exhaustive description of a particular version of the said data type in the
\emph{data structure description language} (DSDL).

A data type definition contains an ordered, possibly empty collection of \emph{field attributes} and/or
unordered, possibly empty collection of \emph{constant attributes}.

A data type may describe either a \emph{structure object} or a \emph{tagged union object}.
The value of a structure object is a function of the values of all of its field attributes.
A tagged union object is formed from at least two field attributes,
but it is capable of holding exactly one field attribute value at any given time.
The value of a tagged union object is a function of which field attribute value
it is holding at the moment and the value of said field attribute.

A field attribute represents a named dynamically assigned value of a statically defined type
that can be exchanged over the network as a member of its containing object.
A padding field attribute is a special kind of field attribute which is used for data alignment purposes;
such field attributes are not named.

A constant attribute represents a named statically defined value of a statically defined type.
Constants are never exchanged over the network, since they are assumed to be known to all involved nodes
by virtue of them sharing compatible definitions of the data type.

Constant values are defined via \emph{DSDL expressions},
which are evaluated at the time of DSDL definition processing.
There is a special category of types called \emph{expression types},
instances of which are used only during expression evaluation
and cannot be exchanged over the network.

Data type definitions can also contain various auxiliary elements reviewed later,
such as deprecation markers (notifying its users that the data type is no longer recommended for new designs)
or assertions (special statements introduced by data type designers
which are statically validated by DSDL processing tools).

Service type definitions are a special case:
they cannot be instantiated or serialized, they do not contain attributes,
and they are composed of exactly two inner data type definitions\footnote{
    A service type can be thought of as a specialized namespace that contains two types and
    has some of the properties of a type, such as name and version.
}.
These inner types are the service request type and the service response type,
separated by the \emph{service response marker}.
They are otherwise ordinary data types except that they are unutterable\footnote{%
    Cannot be referred to. Another commonly used term is ``Voldemort type''.
}
and they derive some of their properties\footnote{Like version numbers or deprecation status.}
from their \emph{parent service type}.

\subsection{Serialization}

Every object that can be exchanged between Cyphal nodes has a well-defined \emph{serialized representation}.
The value and meaning of an object can be unambiguously recovered from its serialized representation,
provided that the type of the object is known.
Such recovery process is called \emph{deserialization}.

\label{sec:dsdl_bit_length_set}
A serialized representation is a sequence of binary digits (bits);
the number of bits in a serialized representation is called its \emph{bit length}.
A \emph{bit length set} of a data type refers to the set of bit length values of all possible
serialized representations of objects that are instances of the data type.

A data type whose bit length set contains more than one element is said to be \emph{variable length}.
The opposite case is referred to as \emph{fixed length}.

The data type of a serialized message or service object exchanged over the network
is recovered from its subject-ID or service-ID, respectively,
which is attached to the serialized object, along with other metadata, in a manner dictated by the applicable
transport layer specification (chapter~\ref{sec:transport}).
For more information on port identifiers and data type mapping refer to section~\ref{sec:basic_subjects_and_services}.

The bit length set is not defined on service types (only on their request and response types)
because they cannot be instantiated.

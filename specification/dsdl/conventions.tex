\section{Conventions and recommendations}

This section is dedicated to conventions and recommendations
intended to help data type designers maintain a consistent style across the ecosystem
and avoid some common pitfalls.
All of the conventions and recommendations provided in this section are optional (not mandatory to follow).

\subsection{Naming recommendations}

The DSDL naming recommendations follow those that are widely accepted in the general software development industry.

\begin{itemize}
    \item Namespaces and field attributes should be named in the \verb|snake_case|.
    \item Constant attributes should be named in the \verb|SCREAMING_SNAKE_CASE|.
    \item Data types (excluding their namespaces) should be named in the \verb|PascalCase|.
    \item Names of message types should form a declarative phrase or a noun. For example,
          \verb|BatteryStatus| or \verb|OutgoingPacket|.
    \item Names of service types should form an imperative phrase or a verb. For example,
          \verb|GetInfo| or \verb|HandleIncomingPacket|.
    \item Short names, unnecessary abbreviations, and uncommon acronyms should be avoided.
\end{itemize}

\subsection{Comments}

Every data type definition file should begin with a header comment that provides an exhaustive description
of the data type, its purpose, semantics, usage patterns, any related data exchange patterns,
assumptions, constraints, and all other information that may be necessary or generally useful for the usage of the
data type definition.

Every attribute of the data type definition, and especially every field attribute of it,
should have an associated comment explaining the purpose of the attribute, its semantics, usage patterns,
assumptions, constraints, and any other pertinent information.
Exception applies to attributes supplied with sufficiently descriptive and unambiguous names.

A comment should be placed after the entity it is intended to describe;
either on the same line (in which case it should be separated from the preceding text with at least two spaces)
or on the next line (without blank lines in between).
This recommendation does not apply to the file header comment.

% Field comment placement https://forum.opencyphal.org/t/dsdl-documentation-comments/407

\subsection{Optional value representation}

Data structures may include optional field attributes that are not always populated.

The recommended approach for representing optional field attributes
is to use variable-length arrays with the capacity of one element.

Alternatively, such one-element variable-length arrays can be replaced with two-field unions,
where the first field is empty and the second field contains the desired optional value.
The described layout is semantically compatible with the one-element array described above,
provided that the field attributes are not swapped.

Floating-point-typed field attributes may be assigned the value of not-a-number (NaN) per IEEE 754
to indicate that the value is not specified;
however, this pattern is discouraged because the value would still have to be transferred over the bus
even if not populated, and special case values undermine type safety.

\begin{remark}[breakable]
    Array-based optional field:

    \begin{minted}{python}
        MyType[<=1] optional_field
    \end{minted}

    Union-based optional field:

    \begin{minted}{python}
        @sealed                         # Sic!
        @union                          # The implicit tag is one byte long.
        uavcan.primitive.Empty none     # Represents lack of value, unpopulated field.
        MyType some                     # The field of interest; field ordering is important.
    \end{minted}

    The defined above union can be used as follows (suppose it is named \verb|MaybeMyType|):

    \begin{minted}{python}
        MaybeMyType optional_field
    \end{minted}

    The shown approaches are semantically compatible.
\end{remark}

\begin{remark}[breakable]
    The implicit truncation and the implicit zero extension rules allow one to freely add such optional fields
    at the end of a definition while retaining semantic compatibility.
    The implicit truncation rule will render them invisible to nodes that utilize older data type definitions
    which do not contain them, whereas nodes that utilize newer definitions will be able to correctly process
    objects serialized using older definitions because the implicit zero extension rule guarantees
    that the optional fields will appear unpopulated.

    For example, let the following be the old message definition:

    \begin{minted}{python}
        float64 foo
        float32 bar
    \end{minted}

    The new message definition with the new field is as follows:

    \begin{minted}{python}
        float64 foo
        float32 bar
        MyType[<=1] my_new_field
    \end{minted}

    Suppose that one node is publishing a message using the old definition,
    and another node is receiving it using the new definition.
    The implicit zero extension rule guarantees that the optional field array will
    appear empty to the receiving node because the implicit length field will be read as zero.
    Same is true if the message was nested inside another one, thanks to the delimiter header.
\end{remark}

\subsection{Bit flag representation}

The recommended approach to defining a set of bit flags is to dedicate a \verb|bool|-typed field attribute for each.
Representations based on an integer sum of powers of two\footnote{Which are popular in programming.}
are discouraged due to their obscurity and failure to express the intent clearly.

\begin{remark}
    Recommended approach:

    \begin{minted}{python}
        void5
        bool flag_foo
        bool flag_bar
        bool flag_baz
    \end{minted}

    Not recommended:

    \begin{minted}{python}
        uint8 flags             # Not recommended
        uint8 FLAG_BAZ = 1
        uint8 FLAG_BAR = 2
        uint8 FLAG_FOO = 4
    \end{minted}
\end{remark}

\section{Attributes}\label{sec:dsdl_attributes}

An \emph{attribute} is a named (excepting padding fields) entity associated with a particular object or type.

\subsection{Composite type attributes}

A composite type attribute that has a value assigned at the data type definition time is called a
\emph{constant attribute};
a composite type attribute that does not have a value assigned at the definition time is called a
\emph{field attribute}.

The name of a composite type attribute shall be unique within the data type definition that contains it,
and it shall not match any of the reserved name patterns specified in the table
\ref{table:dsdl_reserved_word_patterns}.
This requirement does not apply to padding fields.

\subsubsection{Field attributes}

A field attribute represents a named dynamically assigned value of a statically defined type
that can be exchanged over the network as a member of its containing object.
The data type of a field attribute shall be of the serializable type category
(section~\ref{sec:dsdl_serializable_types}),
excepting the void type category, which is not allowed.

Exception applies to the special kind of field attributes --- \emph{padding fields}.
The type of a padding field attribute shall be of the void category.
A padding field attribute may not have a name.

A pair of field attributes is considered equal if, and only if, both field attributes are of the same type,
and both share the same name or both are padding field attributes.

\begin{remark}
    Example:
    \begin{minted}{python}
        uint8[<=10] regular_field   # A field named "regular field"
        void16                      # A padding field; no name is permitted
    \end{minted}
\end{remark}

\subsubsection{Constant attributes}

A constant attribute represents a named statically assigned value of a statically defined type.
Values of constant attributes are never exchanged over the network,
since they are assumed to be known to all involved nodes
by virtue of them sharing the same definition of the data type.

The data type of a constant attribute shall be of the primitive type category
(section~\ref{sec:dsdl_serializable_types}).

The value of the constant attribute is determined at the DSDL definition processing time by evaluating its
\emph{initialization expression}.
The expression shall yield a compatible type upon its evaluation in order to initialize
the value of its constant attribute.
The set of compatible types depends on the type of the initialized constant attribute,
as specified in table~\ref{table:dsdl_constant_init_pattern}.

\begin{CyphalSimpleTable}[wide]{Permitted constant attribute value initialization patterns}{|l | X | X[2] | X[2]|}
    \diagbox[font=\footnotesize]{Constant\\type\\category}{Expression\\type} &
    \texttt{bool} & \texttt{rational} & \texttt{string} \\

    \textbf{Boolean} &
    Allowed. &
    Not allowed. &
    Not allowed. \\

    \textbf{Integer} &
    Not allowed. &
    Allowed if the denominator equals one and the numerator value is within the range of the constant type. &
    Allowed if the target type is \texttt{uint8} and the source string contains one symbol whose code point falls
    into the range $[0, 127]$. \\

    \textbf{Floating point} &
    Not allowed. &
    Allowed if the source value does not exceed the finite range of the constant type.
    The final value is computed as the quotient of the numerator and the denominator
    with implementation-defined accuracy. &
    Not allowed. \label{table:dsdl_constant_init_pattern}\\

\end{CyphalSimpleTable}

Due to the value of a constant attribute being defined at the data type definition time,
the cast mode of primitive-typed constants has no observable effect.

\begin{remark}
    A real literal \verb|1234.5678| is represented internally as
    $\frac{6172839}{5000}$, which can be used to initialize a \verb|float16| value,
    resulting in $1235.0$.

    The specification states that the value of a floating-point constant should be computed
    with an implementation-defined accuracy. Cyphal avoids strict accuracy requirements in order to
    ensure compatibility with implementations that rely on non-standard floating point formats.
    Such laxity in the specification is considered acceptable since the uncertainty is always
    confined to a single division expression per constant; all preceding computations, if any,
    are always performed by the DSDL compiler using exact rational arithmetic.
\end{remark}

\subsection{Local attributes}\label{sec:dsdl_local_attributes}

Local attributes are available at the DSDL definition processing time.

As defined in section~\ref{sec:dsdl_grammar},
a DSDL definition is an ordered collection of statements;
a statement may contain DSDL expressions.
An expression contained in a statement number $E$ may refer to a
composite type attribute introduced in a statement number $A$ by its name,
where $A < E$ and both statements belong to the same data type definition\footnote{
    Per \ref{sec:dsdl_elements_of_data_type_definition},
    in case of services, this applies only to their request and response types.
}.
The representation of the referred attribute in the context of the referring DSDL expression
is specified in table~\ref{table:dsdl_local_attribute_representation}.

\begin{CyphalSimpleTable}{Local attribute representation}{|l X X|}\label{table:dsdl_local_attribute_representation}%
    Attribute category & Value type & Value \\

    Constant attribute &
    Type of the constant attribute &
    Value of the constant attribute \\

%    Field attribute &
%    \texttt{metaserializable} &
%    Type of the field attribute \\
    Field attribute &
    Illegal &
    Illegal \\

\end{CyphalSimpleTable}

\begin{remark}
    \begin{minted}{python}
        uint8 FOO = 123
        uint16 BAR = FOO ** 2
        @assert BAR == 15129
        ---  # The request type ends here; its attributes are no longer accessible.
        #uint16 BAZ = BAR  # Would fail - BAR is not accessible here.
        float64 FOO = 3.14
        @assert FOO == 3.14
    \end{minted}
\end{remark}

\subsection{Intrinsic attributes}

Intrinsic attributes are available in any expression.
Their values are constructed by the DSDL processing tool depending on the context,
as specified in this section.

\subsubsection{Offset attribute}

The offset attribute is referred to by its identifier ``\verb|_offset_|''.
Its value is of type $\texttt{set}_\texttt{<rational>}$.

In the following text, the term \emph{referring statement} denotes a statement
containing an expression which refers to the offset attribute.
The term \emph{bit length set} is defined in section~\ref{sec:dsdl_bit_length_set}.

The value of the attribute is a function of the field attribute declarations preceding the referring statement
and the category of the containing definition.

If the current definition belongs to the tagged union category,
the referring statement shall be located after the last field attribute definition.
A field attribute definition following the referring statement renders the current definition invalid.
For tagged unions, the value of the offset attribute is defined as the
cumulative bit length set\footnote{Section \ref{sec:dsdl_composite_alignment_cumulative_bls}.}
of the union's fields, where each element of the set is incremented by the bit length of the implicit union tag field
(section \ref{sec:dsdl_serialization_composite}).

If the current data definition does not belong to the tagged union category,
the referring statement may be located anywhere within the current definition.
The value of the offset attribute is defined as the
cumulative bit length set\footnote{Section \ref{sec:dsdl_composite_alignment_cumulative_bls}.}
of the fields defined in statements preceding the referring statement
(see section~\ref{sec:dsdl_grammar} on statement ordering).

\begin{remark}
    \begin{minted}{python}
        @union
        uint8 a
        #@print _offset_  # Would fail: it's a tagged union, _offset_ is undefined until after the last field
        uint16 b
        @assert _offset_ == {8 + 8,  8 + 16}
        ---
        @assert _offset_ == {0}
        float16 a
        @assert _offset_ == {16}
        void4
        @assert _offset_ == {20}
        int4 b
        @assert _offset_ == {24}
        uint8[<4] c
        @assert _offset_ == 8 + {24,  32,  40,  48}
        @assert _offset_ % 8 == {0}
        # One of the main usages for _offset_ is statically proving that the following field is byte-aligned
        # for all possible valid serialized representations of the preceding fields. It is done by computing
        # a remainder as shown above. If the field is aligned, the remainder set will equal {0}. If the
        # remainder set contains other elements, the field may be misaligned under some circumstances.
        # If the remainder set does not contain zero, the field is never aligned.
        uint8 well_aligned   # Proven to be byte-aligned.
    \end{minted}
\end{remark}

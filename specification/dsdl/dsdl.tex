\chapter{Data structure description language}\label{sec:dsdl}

\section{General principles}

The data structure description language, or \emph{DSDL}, is a simple domain-specific language designed for
defining compound data types.
The defined data types are used for exchanging data between UAVCAN nodes via one of the standard UAVCAN
transport protocols\footnote{The standard transport protocols are documented in chapter \ref{sec:transport_layer}.
UAVCAN doesn't prohibit users from defining their own application-specific transports as well,
although users doing that are likely to encounter compatibility issues and possibly a suboptimal
performance of the protocol.}.

In accordance with the UAVCAN architecture, DSDL allows users to define data types of two kinds:
message types and service types.
Message types are used to exchange data over publish-subscribe one-to-many message links identified by subject-ID,
and service types are used to perform request-response one-to-one exchanges (like RPC) identified by service-ID.
A message data type includes one data structure which forms the message body,
and a service data type includes two data structures: one of them is used for service request transfers
(from client to server), and the other is used for response transfers (from the server back to the client).

Following the deterministic nature of UAVCAN, the size of all data structures is bounded and statically known.
Variable-size entities always have a fixed size limit defined by the data type designer.

DSDL provides well-defined means of data type versioning, which enable data type maintainers to introduce changes
to released data types while ensuring backward compatibility with fielded systems.

DSDL is designed to support extensive static analysis. Important properties of data type definitions such as
backward binary compatibility and data field layouts can be checked and validated by automatic software tools
before the systems utilizing them are fielded.

DSDL definitions can be used to automatically generate serialization (and deserialization) source code
for any data structure in a target programming language.
A tool that is capable of generating serialization code based on a DSDL definition is called a \emph{DSDL compiler}.
More generically, a software tool designed for working with DSDL definitions is called a \emph{DSDL processor}.

\section{Language specification}

\subsection{Structure}

Every data type is located in a namespace.
Namespaces may be included into higher-level namespaces, forming a tree hierarchy.

A namespace that is at the root of the tree hierarchy (i.e., not nested within another one)
is called a \emph{root namespace}.
A namespace that is located inside another namespace is called a \emph{nested namespace}.

A data type is uniquely identified by its namespaces and its \emph{short name}.
The short name of a data type is the name of the type itself excluding the containing namespaces.

A \emph{full name} of a data type consists of its short name and all of the outer namespace names.
The short name and the namespace names included in a full name are called \emph{name components}.
Name components are ordered: the root namespace is always the first component of the name,
followed by the nested namespaces, if there are any, in the order of their nesting;
the short name is always the last component of the full name.
The full name is formed by joining its name components via the ASCII dot character ``\verb|.|'' (ASCII code 46).

A \emph{full namespace} is formed by joining together all name components except the short name.

A \emph{sub-root namespace} is the nested namespace that is nested inside the root namespace.
Data types that reside directly under a root namespace do not have a sub-root namespace.

The name structure is illustrated on the figure \ref{fig:dsdl_data_type_name_structure}.

\begin{figure}[H]
    $$
    \overbrace{
        \underbrace{
            \underbrace{\texttt{\huge{uavcan}}}_{\substack{\text{root} \\ \text{namespace}}}%
            \texttt{\huge{.}}%
            \underbrace{\texttt{\huge{node}}}_{\substack{\text{nested, also} \\ \text{sub-root} \\ \text{namespace}}}%
            \texttt{\huge{.}}%
            \underbrace{\texttt{\huge{port}}}_{\substack{\text{nested} \\ \text{namespace}}}%
        }_{\text{full namespace}}%
        \texttt{\huge{.}}%
        \underbrace{\texttt{\huge{GetInfo}}}_{\text{short name}}
    }^{\text{full name}}
    $$
    \caption{Data type name structure.\label{fig:dsdl_data_type_name_structure}}
\end{figure}

Every data type definition is assigned a major and minor version number pair.
In order to uniquely identify a data type definition, its version numbers must be specified.

\subsection{File hierarchy}

\subsection{Syntax}


\section{Data serialization}\label{sec:dsdl_data_serialization}

The main design principle behind the serialized representations described in this section is
the maximization of compatibility with native representations used by currently existing and
likely future computer microarchitectures.
The goal is to ensure that the serialized representations defined by DSDL match internal data representations of
modern computers, so that, ideally, a typical system will not have to perform any data conversion whatsoever while
exchanging data over a UAVCAN network.

\newcommand{\hugett}[1]{\texttt{\huge{#1}}}

\subsection{General principles}

The smallest atomic data entity is a bit.
Eight bits form one byte;
within the byte, the bits are ordered so that the most significant bit is considered first (0-th index),
and the least significant bit is considered last (7-th index).

Numeric values consisting of multiple bytes are encoded so that the least significant byte is encoded first;
such format is also known as little-endian.

\begin{figure}[H]
    $$
    \overset{\text{bit index}}{%
        \underbrace{%
            \overset{0}{\hugett{0}}\quad
            \overset{1}{\hugett{1}}\quad
            \overset{2}{\hugett{0}}\quad
            \overset{3}{\hugett{1}}\quad
            \overset{4}{\hugett{0}}\quad
            \overset{5}{\hugett{1}}\quad
            \overset{6}{\hugett{0}}\quad
            \overset{7}{\hugett{1}}\quad
        }_{\substack{\text{0-th byte} \\ \text{least significant byte}}}%
    }
    \hugett{\ldots}
    \overset{\text{bit index}}{%
        \underbrace{%
            \overset{8n+0}{\hugett{0}}\quad
            \overset{8n+1}{\hugett{1}}\quad
            \overset{8n+2}{\hugett{0}}\quad
            \overset{8n+3}{\hugett{1}}\quad
            \overset{8n+4}{\hugett{0}}\quad
            \overset{8n+5}{\hugett{1}}\quad
            \overset{8n+6}{\hugett{0}}\quad
            \overset{8n+7}{\hugett{1}}\quad
        }_{\substack{n\text{-th byte} \\ \text{most significant byte}}}%
    }
    $$
    \caption{Bit ordering and grouping.\label{fig:dsdl_serialization_bit_ordering}}
\end{figure}

Upon completion, a bit sequence must be appended with zero (0) bits at the end until the total length of
the bit sequence is an integer multiple of eight (8).
The transport layer may introduce additional padding bytes (as opposed to bits) of arbitrary values,
as described in relevant sections of chapter \ref{sec:transport_layer}.
When a serialized representation is deconstructed, the values of the padding bits must be ignored.

\subsection{Void types}

The serialized representation of a void-typed field attribute is constructed as a sequence of zero bits.
The length of the sequence equals the numeric suffix of the type name.

When a void-typed field attribute is decoded, the values of respective bits are ignored;
in other words, any bit sequence of correct length is a valid encoded representation of a void-typed field attribute.
This behavior facilitates usage of void fields as placeholders for non-void fields
introduced in newer versions of the data type (section \ref{sec:dsdl_versioning}).

\begin{remark}
    The following data schema will be encoded as a sequence of three zero bits $000_2$
    (five trailing padding bits not included):
    \begin{minted}{python}
        void3
    \end{minted}
    The following bit sequences are valid encoded representations of the schema:
    $000_2$,
    $001_2$,
    $010_2$,
    $011_2$,
    $100_2$,
    $101_2$,
    $110_2$,
    $111_2$.
\end{remark}

\subsection{Primitive types}

\subsubsection{Boolean types}

The serialized representation of a value of type \verb|bool| is a single bit.
If the value represents falsity, the value of the bit is zero (0); otherwise, the value of the bit is one (1).

\subsubsection{Unsigned integer types}\label{sec:dsdl_serialized_unsigned_integer}

The serialized representation of an unsigned integer value of length $n$ bits
(which is reflected in the numerical suffix of the data type name)
is constructed as if the number were to be written in base-2 numerical system
with leading zeros preserved so that the total number of binary digits would equal $n$.

\begin{remark}
    The serialized representation of integer 123 of type \verb|uint9| is $001111011_2$.
\end{remark}

\subsubsection{Signed integer types}

The serialized representation of a non-negative value of a signed integer type is constructed as described
in section \ref{sec:dsdl_serialized_unsigned_integer}.

The serialized representation of a negative value of a signed integer type is computed by
applying the following transformation:
$$2^n + x$$
where $n$ is the bit length of the serialized representation
(which is reflected in the numerical suffix of the data type name)
and $x$ is the value whose serialized representation is being constructed.
The result of the transformation is a positive number,
whose serialized representation is then constructed as described in section \ref{sec:dsdl_serialized_unsigned_integer}.

The representation described here is widely known as \emph{two's complement}.

\begin{remark}
    The serialized representation of integer -123 of type \verb|int9| is $110000101_2$.
\end{remark}

\subsubsection{Floating point types}

The serialized representation of floating point types follows the IEEE 754 series of standards as follows:

\begin{itemize}
    \item \verb|float16| --- IEEE 754 binary16;
    \item \verb|float32| --- IEEE 754 binary32;
    \item \verb|float64| --- IEEE 754 binary64.
\end{itemize}

\subsection{Nested data structures}

Nested data structures are serialized directly in-place,
as if their DSDL definition was pasted directly in place of their reference.
No additional prefixes, suffixes, or padding is provided.

\subsection{Fixed size arrays}

Fixed-size arrays are serialized as a plain sequence of items,
with each item serialized independently in place, with no alignment.
No extra data is added.

Essentially, a fixed-size array of size $X$ elements will be serialized exactly in the same way
as a sequence of $X$ fields of the same type in a row.
Hence, the following two data type definitions will have identical binary representation,
the only actual difference being their representation for the application
if automatic code generation is used.

\begin{minted}{python}
AnyType[3] array
\end{minted}

\begin{minted}{python}
AnyType item_0
AnyType item_1
AnyType item_2
\end{minted}

\subsection{Dynamic arrays}

The following two array definitions are equivalent;
the difference is their representation in the DSDL definition for better readability:
\begin{minted}{python}
AnyType[<42] a      # Can contain from 0 to 41 elements
AnyType[<=41] b     # Can contain from 0 to 41 elements
\end{minted}

A dynamic array is serialized as a sequence of serialized items prepended with an unsigned
integer field representing the number of contained items - the \emph{length field}.
The bit width of the length field is a function of the maximum number of items in the array:
$$\lceil{}\log_2 (X + 1)\rceil{}$$
where $X$ is the maximum number of items in the array.
For example, if the maximum number of items is 251, the length field bit width must be 8 bits;
if the maximum number of items is 1, the length field bit width will be just a single bit.

It is recommended to manually align dynamic arrays by prepending them with void fields
so that the first element is byte-aligned, as that enables more efficient serialization and
deserialization.
This recommendation does not need to be followed if the size of the array elements is not
a multiple of eight bits or if the array elements are of variable size themselves
(e.g., a dynamic array of nested types which contain dynamic arrays themselves).

Consider the following definition:

\begin{minted}{python}
void2                       # Padding - not required, provided as an example
AnyType[<42] array          # The length field is 6 bits wide (see the formula)
\end{minted}

If the array contained three elements,
the resulting binary representation would be equivalent to that of the following definition:

\begin{minted}{python}
void2                       # Padding - not required, provided as an example
uint6 array_length          # Set to 3, because the array contains three elements
AnyType item_0
AnyType item_1
AnyType item_2
\end{minted}

\subsection{Unions}

Similar to dynamic arrays, tagged unions are serialized as two subsequent entities:
the union tag followed by the selected field, with no additional data.

The union tag is an unsigned integer, the bit length of which is a function of the number of fields in the union:
$$\lceil{}\log_2 N\rceil{}$$
where N is the number of fields in the union.
The value serialized in the union tag is the index of the selected field.
Field indexes are assigned according to the order in which they are defined in DSDL,
starting from zero;
i.e. the first defined field has the index 0, the second defined field has the index 1, and so on.

Constants are not affected by the union tag.

It is recommended to manually align unions when they are nested into outer data types by
prepending them with void fields so that the elements are byte-aligned,
as that enables more efficient serialization and deserialization.

Consider the following example:

\begin{minted}{python}
@union                  # In this case, the union tag requires 2 bits
uint16  FOO = 42        # A regular constant attribute
uint16  a               # Index 0
uint8   b               # Index 1
float64 c               # Index 2
uint32  BAR = 42        # Another regular constant
\end{minted}

In order to encode the field \verb|b|, which, according to the definition,
has the data type \verb|uint8|, the union tag should be assigned the value 1.
The following structure will have an identical layout:

\begin{minted}{python}
uint2 tag               # Set to 1
uint8 b                 # The actual data
\end{minted}

If the value of \verb|b| was 7, the resulting serialized byte sequence would be (in binary):
$$%
\underbrace{\texttt{01}}_{\text{tag}}%
\overbrace{\texttt{000001 11}}^{\text{field }\texttt{b}}%
\underbrace{\texttt{000000}}_{\text{padding}}%
$$

\subsection{Composite types}

A serialized representation of an object of composite type $T$ is an ordered set of serialized representations of
its field attribute values joined into a bit string.
The ordering of the serialized representations of the field attribute values follows the order
of field attribute declaration of $T$.

Bit strings do not have any implicit entities such as padding or headers.
Data type developers are advised\footnote{But not required.} to manually align field attributes at
byte boundaries using padding field attributes in order to simplify data layouts
and improve the performance of serialization and deserialization routines.

\begin{remark}
    Consider the following data schema definition,
    where the fields are assigned runtime values shown in the comments:

    \begin{minted}{python}
        #                          decimal    bit string            comment
        truncated uint12 first   # +48858     1011_1110_1101_1010   overflow, MSB truncated
        truncated int3   second  # -1         111                   two's complement
        truncated int4   third   # -5         1011                  two's complement
        truncated int2   fourth  # -1         11                    two's complement
        truncated uint4  fifth   # +136       1000_1000             overflow, MSB truncated
    \end{minted}

    It can be seen that the bit layout is rather complicated because the field boundaries do not align with byte
    boundaries, which makes it a good case study.
    The resulting serialized byte sequence is shown below in the base-2 system,
    where bytes (octets) are comma-separated:

    $$
        \overbrace{\hugett{11011010,1110}}^{\texttt{first}}%
        \underbrace{\hugett{111}}_{\texttt{second}}%
        \overbrace{\hugett{1,011}}^{\texttt{third}}%
        \underbrace{\hugett{11}}_{\texttt{fourth}}%
        \overbrace{\hugett{100,0}}^{\texttt{fifth}}%
        \underbrace{\hugett{0000000}}_{\substack{\text{Implicit padding} \\ \text{to 8-bit alignment}}}
    $$
\end{remark}

\section{Data type compatibility and versioning}\label{sec:dsdl_versioning}

\subsection{Rationale}

Data type definitions may evolve over time as they are refined to better address the needs of their applications.
UAVCAN defines a set of rules that allow data type designers to modify and advance their
data type definitions while ensuring backward compatibility and functional safety.

\subsection{Compatibility}

\subsubsection{Bit compatibility}

A data type or schema $A$ is bit-compatible with a data type or schema $B$ if and only if
the set of serialized representations\footnote{The serialization rules are reviewed
in detail in the section \ref{sec:dsdl_data_serialization}.}
of $A$ is a superset of the set of serialized representations of $B$.

$A$ and $B$ are said to be \emph{mutually bit-compatible} if
their sets of serialized representations are equal.

A \emph{variable-length} data type or schema is a serializable data type or schema
whose set of serialized representations contains bit sequences of different lengths.
Conversely, any data type or schema that is not variable-length is \emph{fixed-length}.

\begin{remark}[breakable]
    The following two definitions are mutually bit-compatible:

    \begin{minted}{python}
        uint32 a
        uint32 b
    \end{minted}

    \begin{minted}{python}
        uint64 c
    \end{minted}

    Consider the following example data type definition; assume that its full data type name is
    \verb|demo.Pair|:

    \begin{minted}{python}
        # demo.Pair.1.0
        float16 first
        float16 second
    \end{minted}

    Further, let the following define a data type named \verb|demo.PairVector|:

    \begin{minted}{python}
        # demo.PairVector.1.0
        demo.Pair.1.0[3] vector
    \end{minted}

    Then the following two definitions are bit-compatible:

    \begin{minted}{python}
        demo.PairVector.1.0 pair_vector
    \end{minted}

    \begin{minted}{python}
        float16 first_0     # pair_vector.vector[0].first
        float16 second_0    # pair_vector.vector[0].second
        float16 first_1     # pair_vector.vector[1].first
        float16 second_1    # pair_vector.vector[1].second
        float16 first_2     # pair_vector.vector[2].first
        float16 second_2    # pair_vector.vector[2].second
    \end{minted}

    The latter definition in the example above is a flattened unrolled form of the former definition.
    As such, in that particular example, both definitions can be used interchangeably;
    an object serialized using one definition can be deserialized using the other definition.
    However, it is also possible to construct bit-compatible definitions that are not functionally equivalent:

    \begin{minted}{python}
        float16 a
        float32 b
    \end{minted}

    \begin{minted}{python}
        float32 a
        float16 b
    \end{minted}

    Even though the above definitions are bit-compatible, one cannot be substituted with the other.
    The problem of functional equivalency is addressed by the concept of semantic compatibility,
    explored in the section \ref{sec:dsdl_semantic_compatibility}.

    Complicated scenarios are possible when a bit belonging to a primitive-typed field attribute
    is handed over to a constrained field such as an implicit array length field or an implicit union tag field.
    Some interesting examples are shown in the table \ref{table:dsdl_many_compat},
    together with a set of serialized representation patterns.
    Remember that the bits belonging to void-typed field attributes are ignored during deserialization.

    % Please do not remove the hard placement specifier [H], it is needed to keep tables ordered.
    \begin{table}[H]\caption{Complex bit compatibility examples}\label{table:dsdl_many_compat}
        \begin{tabu}{|l|X|X|X|X|X|}
            \hline
            \rowfont{\bfseries}
            &A                  &  B                &  C                &  D                &  E                 \\
            \hline

            \multirow{2}{*}{\textbf{Definition}}
            &\texttt{void1}     &\texttt{bool x}    &\texttt{void1}     &\texttt{bool x}    &\texttt{bool[<5] a} \\
            &\texttt{bool[<3] a}&\texttt{bool[<3] a}&\texttt{bool[<4] a}&\texttt{bool[<4] a}&                    \\
            \hline

            \multirow{8}{*}{%
                \begin{tabular}[x]{@{}l@{}}\textbf{Serialized}\\\textbf{representations}\end{tabular}%
            }
            &\multicolumn{2}{l|}{\texttt{000   }}   &\multicolumn{2}{l|}{\texttt{000    }}  &\texttt{000     } \\
            &\multicolumn{2}{l|}{\texttt{001a  }}   &\multicolumn{2}{l|}{\texttt{001a   }}  &\texttt{001a    } \\
            &\multicolumn{2}{l|}{\texttt{010aa }}   &\multicolumn{2}{l|}{\texttt{010aa  }}  &\texttt{010aa   } \\
            &\multicolumn{2}{l|}{\texttt{      }}   &\multicolumn{2}{l|}{\texttt{011aaa }}  &\texttt{011aaa  } \\
            &\multicolumn{2}{l|}{\texttt{100   }}   &\multicolumn{2}{l|}{\texttt{100    }}  &\texttt{100aaaa } \\
            &\multicolumn{2}{l|}{\texttt{101a  }}   &\multicolumn{2}{l|}{\texttt{101a   }}  &\texttt{        } \\
            &\multicolumn{2}{l|}{\texttt{110aa }}   &\multicolumn{2}{l|}{\texttt{110aa  }}  &\texttt{        } \\
            &\multicolumn{2}{l|}{\texttt{      }}   &\multicolumn{2}{l|}{\texttt{111aaa }}  &\texttt{        } \\
            \hline

            {\begin{tabular}[x]{@{}l@{}}\textbf{Bit-compatible}\\\textbf{with}\end{tabular}}
            &B                  & A                 & A, B, D           & A, B, C           & \emph{(none)}    \\
            \hline
        \end{tabu}
    \end{table}
\end{remark}

\subsubsection{Semantic compatibility}\label{sec:dsdl_semantic_compatibility}

A data type $A$ is semantically compatible with a data type $B$
if an application that correctly uses $A$ exhibits a functionally equivalent behavior to an application
that correctly uses $B$.
The property of semantic compatibility is commutative.

\begin{remark}[breakable]
    Despite using different binary layouts, the following two definitions are semantically compatible
    and also bit-compatible:

    \begin{minted}{python}
        uint16 FLAG_A = 1
        uint16 FLAG_B = 256
        uint16 flags
    \end{minted}

    \begin{minted}{python}
        uint8 FLAG_A = 1
        uint8 FLAG_B = 1
        uint8 flags_a
        uint8 flags_b
    \end{minted}

    Therefore, the definitions can be used interchangeably.
    It should be noted here that due to different set of field and constant attributes,
    the source code auto-generated from the provided definitions may be not drop-in replaceable,
    requiring changes in the application;
    however, source-code-level application compatibility is orthogonal to data type compatibility.
\end{remark}

\subsection{Versioning}

\subsubsection{General assumptions}

The concept of versioning applies only to composite data types.
As such, unless specifically stated otherwise, every reference to ``data type''
in this section implies a composite data type.

A data type is uniquely identified by its full name,
assuming that every root namespace is uniquely named.
There is one or more versions of every data type.

A data type definition is uniquely identified by its full name and the version number pair.
In other words, there may be multiple definitions of a data type differentiated by their version numbers.

\subsubsection{Versioning principles}

Every data type definition has a pair of version numbers ---
a major version number and a minor version number, following the principles of semantic versioning.

For the purposes of the following definitions, a \emph{release} of a data type definition stands for
the disclosure of the data type definition to its intended users or to the general public,
or for the commencement of usage of the data type definition in a production system.

In order to ensure a deterministic application behavior and ensure a robust migration path
as data type definitions evolve, UAVCAN requires that all data type definitions that share the same
full name and the same major version number must be semantically compatible with each other
and mutually bit-compatible with each other.

The versioning rules do not extend to scenarios where the name of a data type is changed,
because that would essentially construe the release of a new data type,
which relieves its designer from all compatibility requirements.
When a new data type is first released,
the version numbers of its first definition must be assigned ``1.0'' (major 1, minor 0).

In order to ensure predictability and functional safety of applications that leverage UAVCAN,
the standard requires that once a data type definition is released,
its DSDL source text, name, version numbers, fixed port-ID, and other properties cannot undergo any
modifications whatsoever, with the following exceptions:
\begin{itemize}
    \item Whitespace changes of the DSDL source text are allowed,
    excepting string literals and other semantically sensitive contexts.

    \item Comment changes of the DSDL source text are allowed as long as such changes
    do not affect semantic compatibility of the definition.

    \item A deprecation marker directive (section~\ref{sec:dsdl_directives}) can be added or removed\footnote{%
    Removal is useful when a decision to deprecate a data type definition is withdrawn.}.
\end{itemize}
Addition or removal of the fixed port identifier is not permitted after a data type definition
of a particular version is released.

Therefore, substantial modifications of released data types are only possible by releasing
new definitions of the same data type.
If it is desired and possible to keep the same major version number for a new definition of the data type,
the minor version number of the new definition shall be one greater than the newest existing minor version
number before the new definition is introduced.
Otherwise, the major version number shall be incremented by one and the minor version shall be set to zero.

An exception to the above rules applies when the major version number is zero.
Data type definitions bearing the major version number of zero are not subjected to any compatibility requirements.
Released data type definitions with the major version number of zero are permitted to change in arbitrary
ways without any regard for compatibility.
It is recommended, however, to follow the principles of immutability, releasing every subsequent definition
with the minor version number one greater than the newest existing definition.

For any data type, there shall be at most one definition per version.
In other words, there must be exactly one or zero definitions
per combination of data type name and version number pair.

All data types under the same name must be also of the same kind.
In other words, if the first released definition of a data type is of the message kind,
all other versions must also be of the message kind.

\subsubsection{Fixed port identifier assignment constraints}

The following constraints apply to fixed port-ID assignments:
\begin{align*}
    \exists P(x_{a.b})                          &\rightarrow \exists P(x_{a.c})
    &\mid&\ b < c;\ x \in (M \cup S)
    \\
    \exists P(x_{a.b})                          &\rightarrow         P(x_{a.b}) =    P(x_{a.c})
    &\mid&\ b < c;\ x \in (M \cup S)
    \\
    \exists P(x_{a.b}) \land \exists P(x_{c.d}) &\rightarrow         P(x_{a.b}) \neq P(x_{c.d})
    &\mid&\ a \neq c;\ x \in (M \cup S)
    \\
    \exists P(x_{a.b}) \land \exists P(y_{c.d}) &\rightarrow         P(x_{a.b}) \neq P(y_{c.d})
    &\mid&\ x \neq y;\ x \in T;\ y \in T;\ T = \left\{ M, S \right\}
\end{align*}
where $t_{a.b}$ denotes a data type $t$ version $a.b$ ($a$ major, $b$ minor);
$P(t)$ denotes the fixed port-ID (whose existence is optional) of data type $t$;
$M$ is the set of message types, and $S$ is the set of service types.

\subsubsection{Non-fixed port identifier assignment recommendations}

For non-fixed port identifiers
it is recommended to ensure that any port identifier of a given kind (subject or service)
is only used with one major version of one data type.

Sharing a port identifier of a given kind with different major versions of the same data type or different data types
is not recommended even if such different types are bit-compatible and semantically compatible
because it may complicate network maintenance and analysis; additionally, it may confuse automated
network analysis software.

The kind is specified above explicitly due to the fact that the sets of subject-ID and service-ID are orthogonal.
In other words,
the numeric value of a port-ID may refer to different data types if they are of different kinds.

\subsubsection{Data type version selection}

DSDL compilers should compile every available data type version separately,
allowing the application to choose from all available major and minor version combinations.

When emitting a transfer, the major version of the data type is chosen at the discretion of the application.
The minor version should be the newest available one under the chosen major version.

When receiving a transfer, the node deduces which major version of the data type to use
from its port identifier (either fixed or non-fixed).
The minor version should be the newest available one under the deduced major version\footnote{%
Such liberal minor version selection policy poses no compatibility risks since all definitions under the same
major version are compatible with each other.}.

It follows from the above two rules that when a node is responding to a service request,
the major data type version used for the response transfer shall be the same that is used for the request transfer.
The minor versions may differ, which is acceptable due to the major version compatibility requirements.

\begin{remark}[breakable]
    A simple usage example is provided in this intermission.

    Suppose a vendor named ``Sirius Cybernetics Corporation'' is contracted to design a
    cryopod management data bus for a colonial spaceship ``Golgafrincham B-Ark''.
    Having consulted with applicable specifications and standards, an engineer came up with the following
    definition of a cryopod status message type (named \verb|sirius_cyber_corp.b_ark.cryopod.Status|):

    \begin{minted}{python}
        # sirius_cyber_corp.b_ark.cryopod.Status.0.1

        float16 internal_temperature    # [kelvin]
        float16 coolant_temperature     # [kelvin]

        # Status flags in the lower bits
        uint8 FLAG_COOLING_SYSTEM_A_ACTIVE = 1
        uint8 FLAG_COOLING_SYSTEM_B_ACTIVE = 2
        # Error flags in the higher bits
        uint8 FLAG_PSU_MALFUNCTION = 32
        uint8 FLAG_OVERHEATING     = 64
        uint8 FLAG_CRYOBOX_BREACH  = 128
        # Storage for the above defined flags (this is not the recommended practice)
        uint8 flags
    \end{minted}

    The definition is then deployed to the first prototype for initial laboratory testing.
    Since the definition is experimental, the major version number is set to zero in order to signify the
    tentative nature of the definition.
    Suppose that upon completion of the first trials it is identified that the units must track their power consumption
    in real time for each of the three redundant power supplies independently.

    It is easy to see that the amended definition shown below is neither semantically compatible nor bit-compatible
    with the original definition; however, it shares the same major version number of zero, because the backward
    compatibility rules do not apply to zero-versioned data types to allow for low-overhead experimentation
    before the system is deployed and fielded.

    \begin{minted}{python}
        # sirius_cyber_corp.b_ark.cryopod.Status.0.2

        truncated float16 internal_temperature    # [kelvin]
        truncated float16 coolant_temperature     # [kelvin]

        saturated float32 power_consumption_0     # [watt] Power consumption by the redundant PSU 0
        saturated float32 power_consumption_1     # [watt] likewise for PSU 1
        saturated float32 power_consumption_2     # [watt] likewise for PSU 2

        # Status flags in the lower bits
        uint8 FLAG_COOLING_SYSTEM_B_ACTIVE = 1
        uint8 FLAG_COOLING_SYSTEM_A_ACTIVE = 2
        # Error flags in the higher bits
        uint8 FLAG_PSU_MALFUNCTION = 32
        uint8 FLAG_OVERHEATING     = 64
        uint8 FLAG_CRYOBOX_BREACH  = 128
        # Storage for the above defined flags (this is not the recommended practice)
        uint8 flags
    \end{minted}

    The last definition is deemed sufficient and is deployed to the production system
    under the version number of 1.0: \verb|sirius_cyber_corp.b_ark.cryopod.Status.1.0|.

    Having collected empirical data from the fielded systems, the Sirius Cybernetics Corporation has
    identified a shortcoming in the v1.0 definition, which is corrected in an updated definition.
    Since the updated definition, which is shown below, is mutually semantically
    compatible\footnote{The topic of data serialization is explored in detail in the section
    \ref{sec:dsdl_data_serialization}.}
    with v1.0, the major version number is kept the same and the minor version number is incremented by one:

    \begin{minted}{python}
        # sirius_cyber_corp.b_ark.cryopod.Status.1.1

        saturated float16 internal_temperature    # [kelvin]
        saturated float16 coolant_temperature     # [kelvin]

        float32[3] power_consumption    # [watt] Power consumption by the PSU

        # Error flags (this is the recommended practice)
        bool flag_cryobox_breach
        bool flag_overheating
        bool flag_psu_malfunction

        void3   # Reserved for other flags

        # Status flags (this is the recommended practice)
        bool flag_cooling_system_a_active
        bool flag_cooling_system_b_active
    \end{minted}

    Since the definitions v1.0 and v1.1 are mutually bit-compatible and semantically compatible,
    UAVCAN nodes using either of them can successfully interoperate on the same bus.

    Suppose further that at some point a newer version of the cryopod module,
    equipped with better temperature sensors, is released.
    The definition is updated accordingly to use \verb|float32| for the temperature fields instead of \verb|float16|.
    Seeing as that change breaks the bit compatibility,
    the major version number has to be incremented by one,
    and the minor version number has to be reset back to zero:

    \begin{minted}{python}
        # sirius_cyber_corp.b_ark.cryopod.Status.2.0

        float32 internal_temperature    # [kelvin]
        float32 coolant_temperature     # [kelvin]

        float32[3] power_consumption    # [watt] Power consumption by the PSU

        # Error flags (this is the recommended practice)
        bool flag_cryobox_breach
        bool flag_overheating
        bool flag_psu_malfunction

        void3   # Reserved for other flags

        # Status flags (this is the recommended practice)
        bool flag_cooling_system_a_active
        bool flag_cooling_system_b_active
    \end{minted}

    Nodes using v1.0, v1.1, and v2.0 definitions can still coexist on the same network,
    and they can interoperate successfully as long as they all support at least v1.x or v2.x.

    In general, nodes that need to maximize their compatibility are likely to employ all existing major versions of
    each used data type.
    If there are more than one minor versions available, the highest minor version within the major version should
    be used in order to take advantage of the latest changes in the data type definition.
    It is also expected that in certain scenarios some nodes may resort to publishing the same message type
    using different major versions concurrently to circumvent compatibility issues
    (in the example reviewed here that would be v1.1 and v2.0).
\end{remark}

\section{Conventions and recommendations}

% Data type comments
% Field comments https://forum.uavcan.org/t/dsdl-documentation-comments/407


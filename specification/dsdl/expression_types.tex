\section{Expression types}\label{sec:dsdl_expression_types}

Expression types are a special category of data types whose instances can only exist and be operated upon
at the time of DSDL definition processing.
As such, expression types cannot be used to define attributes,
and their instances cannot be exchanged between nodes.

Expression types are used to represent values of constant expressions which are evaluated
when a DSDL definition is processed.
Results of such expressions can be used to define various constant properties,
such as array length boundaries or values of constant attributes.

Expression types are specified in this section.
Each expression type has a formal DSDL name for completeness;
even if such types can't be used to define attributes,
a well-defined formal name allows DSDL processing tools to emit well-formed
and understandable diagnostic messages.

\subsection{Rational number}\label{sec:dsdl_rational}

At the time of DSDL definition processing, integer and real numbers are represented internally as rational numbers
where the range of numerator and denominator is unlimited\footnote{%
    Technically, the range may only be limited by the memory resources available to the DSDL processing tool.
}.
DSDL processing tools are not permitted to introduce any implicit rational number transformations that
may result in a loss of information.

The DSDL name of the rational number type is ``\verb|rational|''.

Rational numbers are assumed to be stored in a normalized form, where the denominator is positive
and the greatest common divisor of the numerator and the denominator is one.

A rational number can be used in a context where an integer value is expected only if its denominator equals one.

Implicit conversions between boolean-valued entities and rational numbers are not allowed.

\begin{CyphalSimpleTable}{Operators defined on instances of rational numbers}{|l l X X|}
    Op & Type & Constraints & Description
    \label{table:dsdl_operators_rational} \\

    \texttt{\textbf{+}}     & $(\texttt{rational}) \rightarrow \texttt{rational}$ & &
    No effect. \\
    \texttt{\textbf{-}}     & $(\texttt{rational}) \rightarrow \texttt{rational}$ & &
    Negation. \\

    \texttt{\textbf{**}}    & $(\texttt{rational}, \texttt{rational}) \rightarrow \texttt{rational}$ &
    Power denominator equals one &
    Exact exponentiation. \\

    \texttt{\textbf{**}}    & $(\texttt{rational}, \texttt{rational}) \rightarrow \texttt{rational}$ &
    Power denominator greater than one &
    Exponentiation with imp\-lem\-en\-ta\-ti\-on-de\-fin\-ed accuracy. \\

    \texttt{\textbf{*}}  & $(\texttt{rational}, \texttt{rational}) \rightarrow \texttt{rational}$ & &
    Exact multiplication. \\

    \texttt{\textbf{/}}  & $(\texttt{rational}, \texttt{rational}) \rightarrow \texttt{rational}$ &
    Non-zero divisor &
    Exact division. \\

    \texttt{\textbf{\%}} & $(\texttt{rational}, \texttt{rational}) \rightarrow \texttt{rational}$ &
    Non-zero divisor &
    Exact modulo. \\

    \texttt{\textbf{+}}  & $(\texttt{rational}, \texttt{rational}) \rightarrow \texttt{rational}$ & &
    Exact addition. \\

    \texttt{\textbf{-}}  & $(\texttt{rational}, \texttt{rational}) \rightarrow \texttt{rational}$ & &
    Exact subtraction. \\

    \texttt{\textbf{|}}  & $(\texttt{rational}, \texttt{rational}) \rightarrow \texttt{rational}$ &
    Denominators equal one &
    Bitwise or. \\

    \texttt{\textbf{\textasciicircum{}}} & $(\texttt{rational}, \texttt{rational}) \rightarrow \texttt{rational}$ &
    Denominators equal one &
    Bitwise xor. \\

    \texttt{\textbf{\&}} & $(\texttt{rational}, \texttt{rational}) \rightarrow \texttt{rational}$ &
    Denominators equal one &
    Bitwise and. \\

    \texttt{\textbf{!=}} & $(\texttt{rational}, \texttt{rational}) \rightarrow \texttt{bool}$ & & Exact inequality. \\
    \texttt{\textbf{==}} & $(\texttt{rational}, \texttt{rational}) \rightarrow \texttt{bool}$ & & Exact equality. \\
    \texttt{\textbf{<=}} & $(\texttt{rational}, \texttt{rational}) \rightarrow \texttt{bool}$ & & Less or equal. \\
    \texttt{\textbf{>=}} & $(\texttt{rational}, \texttt{rational}) \rightarrow \texttt{bool}$ & & Greater or equal. \\
    \texttt{\textbf{<}}  & $(\texttt{rational}, \texttt{rational}) \rightarrow \texttt{bool}$ & & Strictly less. \\
    \texttt{\textbf{>}}  & $(\texttt{rational}, \texttt{rational}) \rightarrow \texttt{bool}$ & & Strictly greater. \\

\end{CyphalSimpleTable}

\subsection{Unicode string}\label{sec:dsdl_string}

This type contains a sequence of Unicode characters.
It is used to represent string literals internally.

The DSDL name of the Unicode string type is ``\verb|string|''.

A Unicode string containing one symbol whose code point is within $[0, 127]$
(i.e., an ASCII character) is implicitly convertible into a \verb|uint8|-typed constant attribute value,
where the value of the constant is to be equal the code point of the symbol.

\begin{CyphalSimpleTable}{Operators defined on instances of Unicode strings}{|l l X|}
    Op & Type & Description
    \label{table:dsdl_operators_string} \\

    \texttt{\textbf{+}}  & $(\texttt{string}, \texttt{string}) \rightarrow \texttt{string}$ &
    Concatenation. \\

    \texttt{\textbf{!=}} & $(\texttt{string}, \texttt{string}) \rightarrow \texttt{bool}$ &
    Inequality of Unicode NFC normalized forms.
    NFC stands for \emph{Normalization Form Canonical Composition} --
    one of standard Unicode normalization forms where characters are recomposed by canonical equivalence. \\

    \texttt{\textbf{==}} & $(\texttt{string}, \texttt{string}) \rightarrow \texttt{bool}$ &
    Equality of Unicode NFC normalized forms. \\

\end{CyphalSimpleTable}

The set of operations and conversions defined for Unicode strings is to be extended in future versions of
the specification.

\subsection{Set}\label{sec:dsdl_set}

A set type represents an unordered collection of unique objects.
All objects shall be of the same type.
Uniqueness of elements is determined by application of the equality operator ``\verb|==|''.

The DSDL name of the set type is ``\verb|set|''.

A set can be constructed from a set literal, in which case such set shall contain at least one element.

The attributes and operators defined on set instances are listed in the tables~\ref{table:dsdl_set_attributes}
and~\ref{table:dsdl_set_operators}, where $E$ represents the set element type.

\begin{CyphalSimpleTable}{Attributes defined on instances of sets}{|l l X X|}
    Name & Type & Constraints & Description
    \label{table:dsdl_set_attributes} \\

    \texttt{min} & $E$ &
    Operator ``\texttt{<}'' is defined \mbox{$(E, E) \rightarrow \texttt{bool}$} &
    Smallest element in the set determined by sequential application of the operator ``\texttt{<}''. \\

    \texttt{max} & $E$ &
    Operator ``\texttt{<}'' is defined \mbox{$(E, E) \rightarrow \texttt{bool}$} &
    Greatest element in the set determined by sequential application of the operator ``\texttt{<}''. \\

    \texttt{count} & \texttt{rational} & &
    Cardinality. \\

\end{CyphalSimpleTable}

\newcommand\SetElementwiseOperator[1]{%
    \texttt{\textbf{#1}} & $(\texttt{set}_\texttt{<E>}, E) \rightarrow \texttt{set}_\texttt{<R>}$ & $E$ is not a set &
    Elementwise $(E, E) \rightarrow R$.\\

    \texttt{\textbf{#1}} & $(E, \texttt{set}_\texttt{<E>}) \rightarrow \texttt{set}_\texttt{<R>}$ & $E$ is not a set &
    Elementwise $(E, E) \rightarrow R$.\\
}

\begin{CyphalSimpleTable}{Operators defined on instances of sets}{|l l l X|}%
    \label{table:dsdl_set_operators}%
    Op & Type & Constraints & Description \\

    \texttt{\textbf{==}} & $(\texttt{set}_\texttt{<E>}, \texttt{set}_\texttt{<E>}) \rightarrow \texttt{bool}$ & &
    Left equals right. \\

    \texttt{\textbf{!=}} & $(\texttt{set}_\texttt{<E>}, \texttt{set}_\texttt{<E>}) \rightarrow \texttt{bool}$ & &
    Left does not equal right. \\

    \texttt{\textbf{<=}} & $(\texttt{set}_\texttt{<E>}, \texttt{set}_\texttt{<E>}) \rightarrow \texttt{bool}$ & &
    Left is a subset of right. \\

    \texttt{\textbf{>=}} & $(\texttt{set}_\texttt{<E>}, \texttt{set}_\texttt{<E>}) \rightarrow \texttt{bool}$ & &
    Left is a superset of right. \\

    \texttt{\textbf{<}}  & $(\texttt{set}_\texttt{<E>}, \texttt{set}_\texttt{<E>}) \rightarrow \texttt{bool}$ & &
    Left is a proper subset of right. \\

    \texttt{\textbf{>}}  & $(\texttt{set}_\texttt{<E>}, \texttt{set}_\texttt{<E>}) \rightarrow \texttt{bool}$ & &
    Left is a proper superset of right. \\

    \texttt{\textbf{|}} &
    $(\texttt{set}_\texttt{<E>}, \texttt{set}_\texttt{<E>}) \rightarrow \texttt{set}_\texttt{<E>}$ & &
    Union. \\

    \texttt{\textbf{\textasciicircum{}}} &
    $(\texttt{set}_\texttt{<E>}, \texttt{set}_\texttt{<E>}) \rightarrow \texttt{set}_\texttt{<E>}$ & &
    Disjunctive union. \\

    \texttt{\textbf{\&}} &
    $(\texttt{set}_\texttt{<E>}, \texttt{set}_\texttt{<E>}) \rightarrow \texttt{set}_\texttt{<E>}$ & &
    Intersection. \\

    \SetElementwiseOperator{**}
    \SetElementwiseOperator{*}
    \SetElementwiseOperator{/}
    \SetElementwiseOperator{\%}
    \SetElementwiseOperator{+}
    \SetElementwiseOperator{-}

\end{CyphalSimpleTable}

\subsection{Serializable metatype}\label{sec:dsdl_metaserializable}

Serializable types (which are reviewed in section~\ref{sec:dsdl_serializable_types})
are instances of the serializable metatype.
This metatype is convenient for expression of various relations and attributes defined on serializable types.

The DSDL name of the serializable metatype is ``\verb|metaserializable|''.

Available attributes are defined on a per-instance basis.

\section{Grammar}

This section contains the formal definition of the DSDL grammar.
Its notation is introduced beforehand.
The meaning of each element of the grammar and their semantics will be explained in the following sections.

\subsection{Notation}

The notation used in the following definition is a variant of the extended Backus--Naur
form\footnote{This notation is a subset of the notation defined in a Python parsing library titled Parsimonious.
Parsimonious is an MIT-licensed software product authored by Erik Rose;
its sources are available at \url{https://github.com/erikrose/parsimonious}.}.
The rule definition patterns are specified in the table \ref{table:dsdl_grammar_definition_notation}.
The text of the formal definition contains comments which begin with an octothorp and last until the end of the line.

\begin{UAVCANSimpleTable}{Notation used in the formal grammar definition}{|l X|}
    \label{table:dsdl_grammar_definition_notation}
    Pattern & Description \\

    \texttt{"korovan"} &
    Denotes a terminal string of ASCII characters.
    The string is case-sensitive. \\

    \emph{(space)} &
    Concatenation.
    E.g., \texttt{korovan paukan excavator} matches a sequence where the specified tokens
    appear in the defined order. \\

    \texttt{korovan / paukan / excavator} &
    Alternatives.
    The leftmost matching alternative is accepted. \\

    \texttt{korovan?} &
    Optional greedy match. \\

    \texttt{paukan*} &
    Zero or more expressions, greedy match. \\

    \texttt{excavator+} &
    One or more expressions, greedy match. \\

    \texttt{\textasciitilde{}r"regex-pattern"} &
    An IEEE POSIX Extended Regular Expression pattern defined between the double quotes.
    The expression operates on the ASCII character set and is always case-sensitive.
    ASCII escape sequences ``\texttt{\textbackslash{}r}'', ``\texttt{\textbackslash{}n}'', and
    ``\texttt{\textbackslash{}t}'' are used to denote ASCII carriage return (code 13),
    line feed (code 10), and tabulation (code 9) characters, respectively. \\

    \texttt{\textasciitilde{}r'regex-pattern'} &
    As above, with single quotes instead of double quotes. \\

    \texttt{(korovan paukan)} &
    Parentheses are used for grouping. \\
\end{UAVCANSimpleTable}

\subsection{Definition}

From the top level down to the expression rule, the grammar is a valid regular grammar,
meaning that it can be parsed using standard regular expressions.

The grammar definition provided here assumes lexerless parsing;
that is, it applies directly to the unprocessed source text of the definition.

\clearpage\inputminted[fontsize=\scriptsize]{python}{dsdl/grammar.parsimonious}

\subsection{Expressions}

Symbols representing operators belong to the ASCII (basic Latin) character set.

Operators of the same precedence level are evaluated from left to right.

The attribute reference operator is a special case: it is defined for an instance of any type
on its left side and an attribute identifier on its right side.
The concept of ``attribute identifier'' is not otherwise manifested in the type system.
The attribute reference operator is not explicitly documented for any data type;
instead, the documentation specifies the set of available attributes for instances of said type,
if there are any.

\begin{UAVCANSimpleTable}{Unary operators}{|l l X|}
    Symbol                             & Precedence & Description \\
    \texttt{\textbf{+}}                         & 3 & Unary plus \\
    \texttt{\textbf{-}} (hyphen-minus)          & 3 & Unary minus \\
    \texttt{\textbf{!}}                         & 8 & Logical not \\
\end{UAVCANSimpleTable}

\begin{UAVCANSimpleTable}{Binary operators}{|l l X|}
    Symbol                                          & Precedence & Description \\
    \texttt{\textbf{.}} (full stop)                          & 1 & Attribute reference
                                                                   (parent object on the left side,
                                                                   attribute identifier on the right side) \\

    \texttt{\textbf{**}}                                     & 2 & Exponentiation
                                                                   (base on the left side, power on the right side) \\

    \texttt{\textbf{*}}                                      & 4 & Multiplication \\
    \texttt{\textbf{/}}                                      & 4 & Division \\
    \texttt{\textbf{\%}}                                     & 4 & Modulo \\

    \texttt{\textbf{+}}                                      & 5 & Addition \\
    \texttt{\textbf{-}} (hyphen-minus)                       & 5 & Subtraction \\

    \texttt{\textbf{|}} (vertical line)                      & 6 & Bitwise or \\
    \texttt{\textbf{\textasciicircum{}}} (circumflex accent) & 6 & Bitwise xor \\
    \texttt{\textbf{\&}}                                     & 6 & Bitwise and \\

    \texttt{\textbf{==}} (dual equals sign)                  & 7 & Equality \\
    \texttt{\textbf{!=}}                                     & 7 & Inequality \\
    \texttt{\textbf{<=}}                                     & 7 & Less or equal \\
    \texttt{\textbf{>=}}                                     & 7 & Greater or equal \\
    \texttt{\textbf{<}}                                      & 7 & Less \\
    \texttt{\textbf{>}}                                      & 7 & Greater \\

    \texttt{\textbf{||}} (dual vertical line)                & 9 & Logical or \\
    \texttt{\textbf{\&\&}}                                   & 9 & Logical and \\
\end{UAVCANSimpleTable}

\section{Expressions}\label{sec:dsdl_expressions}

\subsection{Operators}

The tables contained in this section specify operators available in DSDL expressions.
Symbols representing operators belong to the ASCII (basic Latin) character set.

Operators of the same precedence level are evaluated from left to right.

The attribute reference operator is a special case: it is defined for an instance of any type
on its left side and an attribute identifier on its right side.
The concept of ``attribute identifier'' is not otherwise manifested in the type system.
The attribute reference operator is not explicitly documented for any data type;
instead, the documentation specifies the set of available attributes for instances of said type,
if there are any.

\begin{UAVCANSimpleTable}{Unary operators}{|l l X|}
    Symbol                             & Precedence & Description \\
    \texttt{\textbf{+}}                         & 3 & Unary plus \\
    \texttt{\textbf{-}} (hyphen-minus)          & 3 & Unary minus \\
    \texttt{\textbf{!}}                         & 8 & Logical not \\
\end{UAVCANSimpleTable}

\begin{UAVCANSimpleTable}{Binary operators}{|l l X|}
    Symbol                             & Precedence & Description \\
    \texttt{\textbf{.}} (full stop)             & 1 & Attribute reference (special case) \\

    \texttt{\textbf{**}}                        & 2 & Exponentiation \\

    \texttt{\textbf{*}}                         & 4 & Multiplication \\
    \texttt{\textbf{/}}                         & 4 & Division \\
    \texttt{\textbf{\%}}                        & 4 & Modulo \\

    \texttt{\textbf{+}}                         & 5 & Addition \\
    \texttt{\textbf{-}} (hyphen-minus)          & 5 & Subtraction \\

    \texttt{\textbf{|}} (vertical line)         & 6 & Bitwise or \\
    \texttt{\textbf{\textasciicircum{}}} (circumflex accent) & 6 & Bitwise xor \\
    \texttt{\textbf{\&}}                        & 6 & Bitwise and \\

    \texttt{\textbf{==}} (dual equals sign)     & 7 & Equality \\
    \texttt{\textbf{!=}}                        & 7 & Inequality \\
    \texttt{\textbf{<=}}                        & 7 & Less or equal \\
    \texttt{\textbf{>=}}                        & 7 & Greater or equal \\
    \texttt{\textbf{<}}                         & 7 & Less \\
    \texttt{\textbf{>}}                         & 7 & Greater \\

    \texttt{\textbf{||}} (dual vertical line)   & 9 & Logical or \\
    \texttt{\textbf{\&\&}}                      & 9 & Logical and \\
\end{UAVCANSimpleTable}

% Rational power: best-effort approximation is allowed.

\subsection{Intrinsic identifiers}

% _offset_
